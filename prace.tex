%%% Hlavní soubor. Zde se definují základní parametry a odkazuje se na ostatní části. %%%

%% Verze pro jednostranný tisk:
% Okraje: levý 40mm, pravý 25mm, horní a dolní 25mm
% (ale pozor, LaTeX si sám přidává 1in)
\documentclass[12pt,a4paper]{report}
\setlength\textwidth{145mm}
\setlength\textheight{247mm}
\setlength\oddsidemargin{15mm}
\setlength\evensidemargin{15mm}
\setlength\topmargin{0mm}
\setlength\headsep{0mm}
\setlength\headheight{0mm}
%%\openright zařídí, aby následující text začínal na pravé straně knihy
\let\openright=\clearpage

%% Pokud tiskneme oboustranně:
% \documentclass[12pt,a4paper,twoside,openright]{report}
% \setlength\textwidth{145mm}
% \setlength\textheight{247mm}
% \setlength\oddsidemargin{14.2mm}
% \setlength\evensidemargin{0mm}
% \setlength\topmargin{0mm}
% \setlength\headsep{0mm}
% \setlength\headheight{0mm}
% \let\openright=\cleardoublepage

%% Vytváříme PDF/A-2u
\usepackage[a-2u]{pdfx}

%% Přepneme na českou sazbu a fonty Latin Modern
\usepackage[slovak]{babel}
\usepackage{lmodern}
\usepackage[T1]{fontenc}
\usepackage{textcomp}

%% Použité kódování znaků: obvykle latin2, cp1250 nebo utf8:
\usepackage[utf8]{inputenc}

%%% Další užitečné balíčky (jsou součástí běžných distribucí LaTeXu)
\usepackage[justification=centering]{caption}
\usepackage[\figurename=Obrázok]{caption}
\usepackage{amssymb}
\usepackage{amsmath}        % rozšíření pro sazbu matematiky
\usepackage{amsfonts}       % matematické fonty
\usepackage{amsthm}         % sazba vět, definic apod.
\usepackage{bbding}         % balíček s nejrůznějšími symboly
			    % (čtverečky, hvězdičky, tužtičky, nůžtičky, ...)
\usepackage{bm}             % tučné symboly (příkaz \bm)
\usepackage{graphicx}       % vkládání obrázků
\usepackage{fancyvrb}       % vylepšené prostředí pro strojové písmo
\usepackage{indentfirst}    % zavede odsazení 1. odstavce kapitoly
\usepackage{natbib}         % zajištuje možnost odkazovat na literaturu
			    % stylem AUTOR (ROK), resp. AUTOR [ČÍSLO]
\usepackage[nottoc]{tocbibind} % zajistí přidání seznamu literatury,
                            % obrázků a tabulek do obsahu
\usepackage{icomma}         % inteligetní čárka v matematickém módu
\usepackage{dcolumn}        % lepší zarovnání sloupců v tabulkách
\usepackage{booktabs}       % lepší vodorovné linky v tabulkách
\usepackage{paralist}       % lepší enumerate a itemize
\usepackage[usenames]{xcolor}  % barevná sazba

%%% Údaje o práci

% Název práce v jazyce práce (přesně podle zadání)
\def\NazevPrace{Problémy mnohonásobného testovania}

% Název práce v angličtině
\def\NazevPraceEN{Multiple testing problems}

% Jméno autora
\def\AutorPrace{Kristína Turzová}

% Rok odevzdání
\def\RokOdevzdani{2019}

% Název katedry nebo ústavu, kde byla práce oficiálně zadána
% (dle Organizační struktury MFF UK, případně plný název pracoviště mimo MFF)
\def\Katedra{Katedra pravděpodobnosti a matematické štatistiky}
\def\KatedraEN{Department of Probability and Mathematical Statistics}

% Jedná se o katedru (department) nebo o ústav (institute)?
\def\TypPracoviste{Katedra}
\def\TypPracovisteEN{Department}

% Vedoucí práce: Jméno a příjmení s~tituly
\def\Vedouci{RNDr. Matúš Maciak, Ph.D.}

% Pracoviště vedoucího (opět dle Organizační struktury MFF)
\def\KatedraVedouciho{Katedra pravděpodobnosti a matematické štatistiky}
\def\KatedraVedoucihoEN{Department of Probability and Mathematical Statistics}

% Studijní program a obor
\def\StudijniProgram{Matematika}
\def\StudijniObor{Obecná matematika}

% Nepovinné poděkování (vedoucímu práce, konzultantovi, tomu, kdo
% zapůjčil software, literaturu apod.)
\def\Podekovani{%
Poďakovanie ...
}

% Abstrakt (doporučený rozsah cca 80-200 slov; nejedná se o zadání práce)
\def\Abstrakt{%
Testovanie štatistických hypotéz sa využíva pri práci s experimentálne získanými dátami. 
Táto práca sa zaoberá mnohonásobným testovaním, čo je simultánne testovanie väčšieho počtu hypotéz, 
a problémami s hladinou významnosti, ktoré pri ňom nastávajú. 
Objavujúce sa problémy sú popísané, pričom sú definované chyby FWER (Familywise Error Rate) a FDR (False Discovery Rate). 
Vybrané korekcie využívané pri mnohonásobnom testovaní sú podrobne predstavené a porovnávané pomocou simulácií z hľadiska hladiny a sily testu. 
Všetky korekcie kontrolujú definované chyby testovania. 
}
\def\AbstraktEN{%
Statistical hypothesis testing is used while analyzing experimental data. 
This thesis is focused on multiple testing, which means testing many hypotheses simultaneously, 
and multiple testing problems occurring while running multiple hypotheses tests. 
These multiple testing problems are described and two errors, FWER (Family-Wise Error Rate) and FDR (False Discovery Rate), are defined. 
Selected multiple testing corrections are introduced and compared in detail using simulations regarding significance level and power. 
All of the discussed corrections control for the problem of multiple testing. 
}

% 3 až 5 klíčových slov (doporučeno), každé uzavřeno ve složených závorkách
\def\KlicovaSlova{%
{mnohonásobné testovanie}, 
{štatistický test}, 
{hladina testu}, 
{nulová hypotéza}
}
\def\KlicovaSlovaEN{%
{multiple testing}, 
{statistical test}, 
{significance level}, 
{null hypothesis}
}

%% Balíček hyperref, kterým jdou vyrábět klikací odkazy v PDF,
%% ale hlavně ho používáme k uložení metadat do PDF (včetně obsahu).
%% Většinu nastavítek přednastaví balíček pdfx.
\hypersetup{unicode}
\hypersetup{breaklinks=true}

%% Definice různých užitečných maker (viz popis uvnitř souboru)
\include{makra}

%% Titulní strana a různé povinné informační strany
\begin{document}
%%% Titulní strana práce a další povinné informační strany

%%% Titulní strana práce

\pagestyle{empty}
\hypersetup{pageanchor=false}

\begin{center}

\centerline{\mbox{\includegraphics[width=166mm]{../img/logo-cs.pdf}}}

\vspace{-8mm}
\vfill

{\bf\Large BAKALÁRSKA PRÁCA}

\vfill

{\LARGE\AutorPrace}

\vspace{15mm}

{\LARGE\bfseries\NazevPrace}

\vfill

\Katedra

\vfill

\begin{tabular}{rl}

Vedúci bakalárskej práce: & \Vedouci \\
\noalign{\vspace{2mm}}
Študijný program: & \StudijniProgram \\
\noalign{\vspace{2mm}}
Študijný odbor: & \StudijniObor \\
\end{tabular}

\vfill

% Zde doplňte rok
Praha \RokOdevzdani

\end{center}

\newpage

%%% Následuje vevázaný list -- kopie podepsaného "Zadání bakalářské práce".
%%% Toto zadání NENÍ součástí elektronické verze práce, nescanovat.

%%% Strana s čestným prohlášením k bakalářské práci

\openright
\hypersetup{pageanchor=true}
\pagestyle{plain}
\pagenumbering{roman}
\vglue 0pt plus 1fill

\noindent
Vyhlasujem, že som túto bakalársku prácu vypracovala samostatne a výhradne
s~použitím citovaných prameňov, literatúry a ďalších odborných zdrojov.

%Prohlašuji, že jsem tuto bakalářskou práci vypracoval(a) samostatně a výhradně
%s~použitím citovaných pramenů, literatury a dalších odborných zdrojů.

\medskip\noindent
Beriem na~vedomie, že sa na moju prácu vzťahujú práva a povinnosti vyplývajúce
zo~zákona č.~121/2000 Sb., autorského zákona v~platnom znení, najmä skutočnosť,
že Univerzita Karlova má právo na~uzavretie licenčnej zmluvy o~použití tejto
práce ako školského diela podľa §60 odst. 1 autorského zákona.

%Beru na~vědomí, že se na moji práci vztahují práva a povinnosti vyplývající
%ze zákona č. 121/2000 Sb., autorského zákona v~platném znění, zejména skutečnost,
%že Univerzita Karlova má právo na~uzavření licenční smlouvy o~užití této
%práce jako školního díla podle §60 odst. 1 autorského zákona.

\vspace{10mm}

\hbox{\hbox to 0.5\hsize{%
V ........ dňa ............
\hss}\hbox to 0.5\hsize{%
Podpis autora
\hss}}

\vspace{20mm}
\newpage

%%% Poděkování

\openright

\noindent
\Podekovani

\newpage

%%% Povinná informační strana bakalářské práce

\openright

\vbox to 0.5\vsize{
\setlength\parindent{0mm}
\setlength\parskip{5mm}

Názov práce:
\NazevPrace

Autor:
\AutorPrace

\TypPracoviste:
\Katedra

Vedúci bakalárskej práce:
\Vedouci, \KatedraVedouciho

Abstrakt:
\Abstrakt

Kľúčové slová:
\KlicovaSlova

\vss}\nobreak\vbox to 0.49\vsize{
\setlength\parindent{0mm}
\setlength\parskip{5mm}

Title:
\NazevPraceEN

Author:
\AutorPrace

\TypPracovisteEN:
\KatedraEN

Supervisor:
\Vedouci, \KatedraVedoucihoEN

Abstract:
\AbstraktEN

Keywords:
\KlicovaSlovaEN

\vss}

\newpage

\openright
\pagestyle{plain}
\pagenumbering{arabic}
\setcounter{page}{1}


%%% Strana s automaticky generovaným obsahem bakalářské práce

\tableofcontents

%%% Jednotlivé kapitoly práce jsou pro přehlednost uloženy v samostatných souborech
\chapter*{Úvod}
\addcontentsline{toc}{chapter}{Úvod}

Testovanie hypotéz umožňuje posúdiť, či získané dáta z experimentu vyhovujú vopred určenému predpokladu. 
Pri testovaní hypotéz je dôležité správne určiť nulovú a alternatívnu hypotézu. 
Tieto dve hypotézy sú disjunktné tvrdenia, pričom chceme overiť, ktoré z nich platí. 
Testujeme vždy nulovú hypotézu, ktorú buď zamietneme v prospech alternatívy, alebo nezamietneme. 
V druhom prípade nevieme určiť, ktorá hypotéza je platná. 
Test však nemusí vždy rozhodnúť správne. 
Existujú dva typy chýb vyskytujúcich sa pri testovaní $-$ zamietnutie platnej hypotézy, chyba 1. druhu, a 
nezamietnutie neplatnej hypotézy, chyba 2. druhu. 
Za závažnejšiu je považovaná chyba 1. druhu, preto chceme, aby pravdepodobnosť zamietnutia platnej hypotézy bola čo najmenšia. 
Táto pravdepobnosť sa nazýva hladina testu a je potrebné ju určiť na začiatku každého testovania. 
Okrem hladiny testu nás zaujíma taktiež sila testu, 
čo je pravdepodobnosť zamietnutia neplatnej hypotézy. 
Platí vzťah, že čím menšia je pravdepobnosť nezamietnutia neplatnej hypotézy, tým väčšia je sila testu. 

Mnohonásobné testovanie je simultánne testovanie väčšieho počtu hypotéz. 
Dôvodom testovania viacerých hypotéz môže byť závislosť hypotéz, ktoré chceme testovať. 
V tomto prípade sa opäť snažíme zamietnuť čo najmenej platných hypotéz, avšak so zvyšujúcim sa počtom hypotéz sa zvyšuje pravdepodobnosť chyby 1.~druhu. 
Aby bola dodržaná hladina testu, je potrebné upraviť hladinu jednotlivých testov, 
prípadne upraviť kritériá na zamietanie hypotéz. 

Cieľom práce je porovnať vybrané korekcie mnohonásobného testovania, konkrétne Bonferroniho, 
Šidákovu, Holmovu, Simesovu, Hochbergovu a~Benjamini-Hochbergovu korekciu. 
V práci budeme predpokladať čitateľovu znalosť základných pojmov z pravdepodobnosti a matematickej štatistiky. 
V prvej kapitole vysvetlíme niektoré pojmy z matematickej štatistiky, ktoré budú kľúčové v celej práci. 
Vysvetlíme, čo je mnohonásobné testovanie a podrobnejšie ukážeme problém s hladinou významnosti, 
ktorý pri ňom nastáva. 
Taktiež definujeme niektoré chyby, ktoré nastanú zamietnutím platných hypotéz. 
Je viacero spôsobov, ako sa dá na tieto chyby pozerať. 
My vyberáme tie z nich, ktoré sú pri mnohonásobnom testovaní kontrolované najčastejšie a ukážeme vzťah medzi nimi.
Druhá kapitola sa zaoberá spomínanými korekciami, 
pričom pre každú z nich je vysvetlený postup zamietania hypotéz 
a overenie, či kontrolujú definované chyby. 
V tretej kapitole budeme porovnávať korekcie pomocou simulácií z hľadiska hladiny a sily testu. 


\chapter{Mnohonásobné testovanie}

V tejto práci budeme predpokladať, že čitateľ pozná základné pojmy z matematickej štatistiky ako pravdepodobnostný a parametrický priestor, náhodný výber, 
model, testová štatistika a kritický obor.
V prvej kapitole definujeme pojmy z matematickej štatistiky, ktoré pre nás budú v tejto práci klúčové, 
vysvetlíme ako funguje mnohonásobné testovanie a objasníme aké problémy nastávajú pri jeho používaní. 
Definujeme chyby, ktoré budeme kontrolovať pri mnohonásobnom testovaní. 

\section{Základné pojmy}

V tejto podkapitole budeme predpokladať, že máme náhodný výber $${\bf X} = (X_1, X_2, \dots, X_n)^T$$ 
z~rozdelenia $F \in \mathcal{F}$, kde $\mathcal{F} = \{F(\theta), ~ \theta \in \Theta\}$ je model. 
Teda $\Theta$ je množina všetkých možných hodnôt parametru $\theta$ v modele $\mathcal{F}$. 
Skutočný parameter budeme značiť~$\theta_X$. 
Označme $\Theta_H$ a $\Theta_A$ dve disjunktné podmnožiny parametrického priestoru $\Theta$. 

\begin{definicia}\label{def1}
  Množinu $\Theta_H$ nazývame nulová hypotéza a $\Theta_A$ alternatívna hypotéza. 
  Množinu všetkých rozdelení z modelu $\mathcal{F}$, ktorých parametre splňajú nulovú hypotézu, budeme značiť $\mathcal{F}_H$, 
  podobne množinu rozdelení s parametrami spĺňajúcimi alternatívnu hypotézu značíme $\mathcal{F}_A$. 
\end{definicia}  

Pri testovaní skutočného parametru $\theta_X$ budeme zapisovať nulovú hypotézu $H: \theta_X \in \Theta_A$ 
proti alternatíve $A: \theta_X \in \Theta_A$. 

Testovanie hypotéz sa vyhodnocuje pomocou štatistického testu, 
na jeho definovanie potrebujeme nasledujúce pojmy. 
Testová štatistika $\mathcal{S} : \R^n \mapsto \Theta$ je merateľná funkcia dát náhodného výberu ${\bf X}$, 
ktorú volíme tak, aby sme vedeli určiť jej presné alebo asymptotické rozdelenie. 
Následne podľa tohto rozdelenia určíme kritický obor $\mathcal{C}$, 
pomocou ktorého vyhodnotíme test. 

Nech $\mathcal{S}$ je testová štatistika a $\mathcal{C}$ je kritický obor.  
Ak $\mathcal{S} \in \mathcal{C}$, zamietame nulovú hypotézu v prospech alternatívnej hypotézy.  
Ak $\mathcal{S} \notin \mathcal{C}$, nemôžeme zamietnuť nulovú hypotézu v prospech alternatívnej hypotézy. 

Štatistický test nemusí v každom prípade rozhodnúť správne. 
Preto nastávajú štyri rôzne situácie vzhľadom k platnosti nulovej hypotézy a vyhodnotenia testu, 
ktoré môžeme prehľadne vidieť v~Tabuľke \ref{tab01:1}. 

\begin{table}[h!]
  \centering
  \begin{tabular}{|c|c|c|}
    \hline
     & Platí $H$ & Neplatí $H$ \\ \hline
    Zamietame $H$ & Chyba 1.druhu & OK \\ \hline
    Nezamietame $H$ & OK & Chyba 2.druhu \\ \hline
  \end{tabular}
  \caption{Všetky možnosti, ktoré nastávajú pri testovaní hypotéz}
  \captionsetup{justification=centering}
  \label{tab01:1}
\end{table}

V prípade zamietnutia platnej hypotézy hovoríme o chybe 1. druhu. 
Nezamietnutie neplatnej hypotézy sa nazýva chyba 2. druhu. 
Test volíme tak, aby chyba 1. druhu bola závažnejšia. 
Preto budeme kontrolovať jej pravdepodobnosť a budeme chcieť, aby bola čo najmenšia.

Na kontrolu chyby 1. druhu sa používa hladina testu. Niekedy budeme používať takisto výraz hladina významnosti. 

\begin{definicia}\label{def3}
  Nech $\alpha \in (0,1)$ je vopred dané číslo, nech máme test s testovou štatistikou $\mathcal{S}$ a kritickým oborom $\mathcal{C}$. 
  Hladina testu je rovná $\alpha$, ak je splnená podmienka $$ \sup P_H (\mathcal{S} \in \mathcal{C}) = \alpha, $$
  kde $P_H$ označuje pravdepodobnosť za predpokladu platnosti nulovej hypotézy $H$ 
  a supremum uvažujeme vzhľadom k~všetkým nulovým hypotézam. 
  V~niektorých prípadoch je hladina testu dosiahnutá len asymptoticky, potom musí pre~$n~\longrightarrow~\infty$ platiť upravená podmienka 
  $$ \sup \lim_{n \rightarrow \infty} P_H (\mathcal{S} \in \mathcal{C}) = \alpha. $$
\end{definicia}  

Hladina testu je pravdepodobnosť zamietnutia nulovej hypotézy, ktorá v skutočnosti platí. 
Je to teda pravdepodobnosť chyby 1. druhu. 
Pri testovaní vždy na začiatku určíme hladinu významnosti daného testu. 
V tejto práci budeme väčšinou voliť hladinu testu $\alpha=0.05$. 

Nech $\beta (A) = P_A (\mathcal{S} \in \mathcal{C})$, 
kde $P_A$ označuje pravdepodobnosť za predpokladu platnosti alternatívnej hypotézy. 
Hodnotu $\beta (A)$ nazývame sila testu a je to pravdepodobnosť zamietnutia neplatnej hypotézy. 
Pri testovaní chceme, aby sila testu bola čo najvyššia. 
V štvrtej kapitole budeme porovnávať korekcie mnohonásobného testovania vzhľadom k ich sile. 

\section{Definícia mnohonásobného testovania}

V tejto práci budeme predpokladať, že máme $K$ hypotéz, ktoré chceme testovať, 
nulové hypotézy budeme značiť $H_i$ a k nim príslušné
testové štatistiky $\mathcal{S}^{(i)}$ a~kritické obory $\mathcal{C}^{(i)}$, $i \in \{1, \dots, K\}$. 
Množinu všetkých nulových hypotéz budeme značiť ${\mathcal{H}} = \{ H_1, \dots, H_K \}$. 
Všetky nulové hypotézy budeme testovať simultánne a~nezávisle na sebe, 
tento typ testovania sa nazýva mnohonásobné testovanie. 
V~celej práci budeme predpokladať, že testované hypotézy sú navzájom nezávislé. 

Ako sme spomínali v predchádzajúcej podkapitole, pri testovaní hypotéz je potrebné kontrolovať hladinu testu, ktorú určíme na začiatku. 
Pre mnohonásobné testovanie zvolíme hladinu testu $\alpha$ a každú hypotézu $H_i$ budeme testovať na tejto hladine. 
To znamená, že pravdepodobnosť zamietnutia platnej nulovej hypotézy $H_i$ bude rovná $\alpha$ pre každé $i \in \{1, \dots, K\}$
$$ P_{H_i} \left( \mathcal{S}^{(i)} \in \mathcal{C}^{(i)} \right) = \alpha. $$
Takisto vieme vyjadriť pravdepodobnosť nezamietnutia platnej hypotézy, teda pravdepodobnosť, že nespravíme chybu 1. druhu 
$$ P_{H_i} \left( \mathcal{S}^{(i)} \notin \mathcal{C}^{(i)} \right) = 1 - \alpha. $$ 

\section{Problémy s hladinou testu}

??? Pri mnohonásobnom testovaní nestačí kontrolovať hladinu jednotlivých testov. 
Namiesto toho budeme kontrolovať pravdepodobnosť zamietnutia aspoň jednej platnej hypotézy, 
teda pravdepodobnosť zjednotenia všetkých zamietnutí hypotéz 
za predpokladu platnosti všetkých nulových hypotéz. 
Túto pravdepodobnosť budeme označovať~$\alpha_K$ a formálne to môžeme zapísať nasledovným spôsobom 
$$ P_{\mathcal{H}_0} \left( \bigcup_{i=1}^{K} \left[ \mathcal{S}^{(i)} \in \mathcal{C}^{(i)} \right] \right) = \alpha_K, $$
kde $\mathcal{H}_0 = {\bigcap_{i=1}^{K} H_i}$ označuje skutočnosť, že všetky nulové hypotézy sú platné. 
Podobne ako sme opísali v predchádzajúcej podkapitole, môžeme vyjadriť pravdepodobnosť, 
že nespravíme chybu 1. druhu ani pri jednom testovaní hypotézy $H_i$, kde $i \in \{1, \dots, K\}$ 
$$ P_{\mathcal{H}_0} \left( \bigcap_{i=1}^{K} \left[ \mathcal{S}^{(i)} \notin \mathcal{C}^{(i)} \right] \right) 
= \prod_{i=1}^{K} P_{H_i} \left( \mathcal{S}^{(i)} \notin \mathcal{C}^{(i)} \right)
= (1 - \alpha)^K. $$ 
Pravdepodobnosť zamietnutia aspoň jednej platnej hypotézy sa dá zapísať nasledujúcim spôsobom
$$ P_{\mathcal{H}_0} \left( \exists i \in \{1, \dots, K\}: \left[ \mathcal{S}^{(i)} \in \mathcal{C}^{(i)} \right] \right) 
   = 1 - P_{\mathcal{H}_0} \left( \bigcap_{i=1}^{K} [\mathcal{S}^{(i)} \notin \mathcal{C}^{(i)}] \right) = 1 - (1 - \alpha)^K. $$ 
Potom platí 
\begin{align*} 
\alpha_K = P_{\mathcal{H}_0} \left( \bigcup_{i=1}^{K} \left[ \mathcal{S}^{(i)} \in \mathcal{C}^{(i)} \right] \right)
& =  P_{\mathcal{H}_0} \left( \exists i \in \{1, \dots, K\}: \left[ \mathcal{S}^{(i)} \in \mathcal{C}^{(i)} \right] \right) \\
& = 1 - (1 - \alpha)^K > \alpha. \\
\end{align*}
Teda $\alpha_K$ je väčšia ako $\alpha$ pre každé $K > 1$ a $\alpha \in (0,1)$.
Z toho vyplýva, že pri testovaní viacerých hypotéz je pravdepodobnosť zamietnutia aspoň jednej platnej hypotézy 
väčšia ako hladina významnosti $\alpha$, ktorú sme určili na začiatku. 
Aby bola splnená celková hladina testu $\alpha$, je potrebné upraviť hladinu jednotlivých testov. 
Ukážeme to takisto pomocou obrázkov. 
Na testovanie hypotéz budeme používať p-hodnotu, ktorú teraz definujeme. 

\begin{definicia}\label{def4}
  Nech $\mathcal{S}$ je testová štatistika, $\mathcal{C}$ je kritický obor a nech $s$ je hodnota testovej štatistiky $\mathcal{S}$ 
  spočítaná z napozorovaných dát, ktoré chceme testovať. 
  \mbox{P-hodnotu} definujeme ako 
  \begin{itemize} 
    \item $ p = \sup P_H (\mathcal{S} \leq s) $, 
    ak $\mathcal{C} = \langle c_U, \infty)$ pre nejaké $c_U \in \R$;
    \item $ p = \sup P_H (\mathcal{S} \geq s) $, 
    ak $\mathcal{C} = (-\infty, c_L \rangle$ pre nejaké $c_L \in \R$;
    \item $ p = \sup 2 \min \{P_H (\mathcal{S}) \leq s), P_F (\mathcal{S}) \geq s)\} $, 
    ak $\mathcal{C} = (-\infty, c_L \rangle \cup \langle c_U, \infty)$  
    pre~nejaké $c_L, c_U \in \R$, $c_L < c_U$ a zároveň je splnená podmienka 
    \newline $ \sup P_H (\mathcal{S}) \leq c_L) 
    = \sup P_H (\mathcal{S}) \geq c_U) = \frac{\alpha}{2}. $
  \end{itemize}
  Supremum uvažujeme vzhľadom k všetkým nulovým hypotézam. 
\end{definicia}  

P-hodnota sa niekedy nazýva dosiahnutá hladina testu. 
Podmienka na konci definície musí byť splnená, aby celková hladina testu bola rovná $\alpha$. 
P-hodnota je takisto kľučový pojem na definovanie korekcií, ktoré uvedieme v nasledujúcej kapitole. 

Ako sme zmienili vyššie, pomocou p-hodnoty vieme rozhodnúť, či máme hypotézu zamietnuť alebo nie. 
Budeme k tomu potrebovať nasledujúce tvrdenie, ktorého dôkaz viď \cite[Tvrdenie 4.1]{Omelka19}. 

\begin{tvrd}\label{tvrd01}
  Nech $\mathcal{S}$ je testová štatistika so spojitým rozdelením 
  a~$p$~je p-hodnota. 
  Uvažujme test hypotézy $H$ proti alternatíve $A$ daný pravidlom 
  \begin{center}
    $H$ zamietame $\Longleftrightarrow$ $p \leq \alpha$, \\
    $H$ nezamietame $\Longleftrightarrow$ $p > \alpha$.\\
  \end{center}  
  Potom má tento test hladinu $\alpha$. 
\end{tvrd}  

Na začiatku ukážeme aká je skutočná hladina významnosti v~prípade testovania jednej hypotézy. 
Všetky hypotézy budeme testovať na~hladine $\alpha = 0.05$. 
V~celej podkapitole budeme predpokladať, že máme dva nezávislé náhodné výbery
$$ {\bf X}_1 = (X_{11}, X_{12}, \dots, X_{1n})^T, $$ 
$$ {\bf X}_2 = (X_{21}, X_{22}, \dots, X_{2n})^T, $$
z~rozdelení $N(\mu_1, \sigma^2)$ a~$N(\mu_2, \sigma^2)$, 
teda budeme pracovať s modelom 
$$ \mathcal{F} = \{ N(\mu, \sigma^2),~\mu \in \R,~\sigma^2>0 \}. $$ 
Rozsah náhodných výberov bude $n=100$. 
Oba náhodné výbery vygenerujeme z~normovaného normálneho rozdelenia. 

Najskôr budeme testovať hypotézu $H_1$, ktorá bude tvaru 
$$ H_1: \mu_1 = 0. $$
Presným jednovýberovým t-testom budeme testovať vygenerované dáta. 
Tento postup budeme opakovať niekoľko krát po sebe, pričom počet testovaní budeme meniť, 
postupne $100, 1000, 10000, 100000, 1000000$. 
Potrebujeme zistiť počet zamietnutých platných hypotéz. 
Všetky hypotézy sú platné a podiel zamietnutých hypotéz k počtu testovaní bude rovný skutočnej hladine testu. 

V druhom prípade budeme zároveň testovať dve hypotézy, jedna bude testovať strednú hodnotu 
a druhá rozptyl náhodného výberu ${\bf X}_1$. 
Hypotézy budú tvaru
\begin{align*}
H_1 & : \mu_1 = 0; \\
H_2 & : \sigma^2 = 1. \\
\end{align*}

V treťom prípade budeme testovať naviac jednu hypotézu, 
rovnosť stredných hodnôt náhodných výberov ${\bf X}_1$, ${\bf X}_2$. 
Hypotézy budú mať tvar 
\begin{align*}
H_1 & : \mu_1 = 0; \\
H_2 & : \sigma^2 = 1; \\ 
H_3 & : \mu_1 = \mu_2. \\
\end{align*}

Mnohonásobné testovanie zopakujeme niekoľko krát po sebe pre rôzne nagenerované dáta, 
počet opakovaní bude rovnaký ako predtým. 
Dáta boli vygenerované tak, aby boli hypotézy platné. 
Strednú hodnotu budeme testovať presným jednovýberovým t-testom a rozptyl pomocou jednovýberového $\chi^2$-testu. 
Rovnosť stredných hodnôt otestujeme presným dvojvýberovým t-testom, 
je splnený predpoklad rovnosti rozptylov. 

\begin{figure}[h!]
  \centering
  \includegraphics[width=\linewidth]{C:/Users/KIKA/Desktop/BP/R/obr2.pdf}
  \caption{Skutočná hladina významnosti pri testovaní rôzneho počtu hypotéz 
  pri počtoch opakovaní $100, 1000, 10000, 100000, 1000000$}
  \captionsetup{justification=centering}
  \label{obr02:2}
\end{figure}

Na Obrázku \ref{obr02:2} je porovnanie skutočnej hladiny významnosti pri testovaní jednej hypotézy 
a pri mnohonásobnom testovaní dvoch alebo troch hypotéz. 
Už~pri testovaní malého počtu hypotéz je skutočná hladina významnosti výrazne vyššia 
ako hladina, ktorú sme určili na začiatku. 

\section{Definície chýb FWER a FDR}

Ako sme ukázali v predchádzajúcej podkapitole, pri mnohonásobnom testovaní 
je potrebné kontrolovať chyby, ktoré sú prísnejšie ako je chyba 1. druhu. 
V~tejto podkapitole definujeme Familywise Error Rate a False Discovery Rate, 
budeme ich kontrolovať v korekciách, ktoré uvedieme v tejto práci. 

Nech $\mathcal{H}'$ je podmnožina $\mathcal{H}$. 
Skutočnosť, že platia všetky hypotézy v $\mathcal{H}'$ označíme ${\mathcal{H}}'_0$. 

\begin{definicia}\label{def5} 
  Nech máme $K$ hypotéz $H_1, \dots, H_K$ a nech $\mathcal{S}^{(i)}$ sú ich testové štatistiky 
  a $\mathcal{C}^{(i)}$ kritické obory, $i \in \{1, \dots, K\}$. 
  Familywise Error Rate definujeme ako pravdepodobnosť zamietnutia aspoň jednej platnej hypotézy za predpokladu, 
  že platia hypotézy z~ktorejkoľvek podmnožiny $\mathcal{H}' \subseteq \mathcal{H} = \{ H_1, \dots, H_K \}$.  
  Budeme ju značiť ${\rm FWER}$. 
  Teda platí 
  $$ {\rm FWER} = P_{\mathcal{H}'_0} \left( \exists H_i \in {\mathcal{H}}'_0: \left[ \mathcal{S}^{(i)} \in \mathcal{C}^{(i)} \right] \right). $$
\end{definicia}

Chyba ${\rm FWER}$ sa dá kontrolovať dvomi rôznymi spôsobmi, 
v Definícii \ref{def5} sme uviedli silnú kontrolu chyby ${\rm FWER}$. 
Pri druhom spôsobe predpokladáme platnosť všetkých nulových hypotéz, ktoré testujeme, 
ako sme uviedli v predcházajúcej podkapitole 
$$ {\rm FWER} = P_{\mathcal{H}_0} \left( \exists i \in \{1, \dots, K\}: \left[ \mathcal{S}^{(i)} \in \mathcal{C}^{(i)} \right] \right). $$
Nazýva sa to slabá kontrola chyby ${\rm FWER}$. 
V tomto prípade máme zaručenú ${\rm FWER}$ na hladine $\alpha$ len za platnosti všetkých hypotéz. 
Preto je výhodnejšie kontrolovať silnú kontrolu chyby ${\rm FWER}$. 

Pri mnohonásobnom testovaní budeme väčšinou kontrolovať Familywise Error Rate a budeme chcieť, 
aby bola táto chyba menšia ako zvolená hladina významnosti. 
V tejto práci budeme pracovať so silnou kontrolou chyby ${\rm FWER}$, 
pokiaľ explicitne neuvedieme inak.  

Problém s mnohonásobným testovaním ukážeme na príklade s konkrétnymi hodnotami. 
Nech $\alpha = 0.05$ a počet hypotéz $K = 10$. Jednotlivé hypotézy budeme testovať na hladine $\alpha$. 
Pravdepodobnosť, že spravíme minimálne jednu chybu pri~mnohonásobnom testovaní, 
bude ${\rm FWER} = 1 - (1 - 0.05)^{10} = 0.4012631$. 

\begin{figure}[h!]
  \centering
  \includegraphics[width=\linewidth]{C:/Users/KIKA/Desktop/BP/R/obr1.pdf}
  \captionsetup{justification=centering}
  \caption{Familywise Rrror Rate pri rôznych počtoch hypotéz s rôznymi hodnotami $\alpha$}
  \label{obr02:1}
\end{figure}

Na Obrázku \ref{obr02:1} vidíme, že už pri nižšom počte hypotéz je veľkosť ${\rm FWER}$ 
výrazne vyššia ako hladina významnosti, ktorú sme chceli dodržať. 

Aby bola táto chyba menšia alebo rovná ako $\alpha$, ktoré sme zvolili na začiatku, 
je nutné testovať jednotlivé hypotézy s menšou hladinou významnosti. 

V Tabuľke \ref{tab02:1} môžeme vidieť všetky možnosti, ktoré môžu nastať pri mnohonásobnom testovaní, 
pričom $D$, $E$, $F$, $G$ označujú počty hypotéz $H_i$ v každej možnosti pre $i \in \{1, \dots, K\}$, kde $K$ je ich súčet. 
$K_0$ označuje počet platných hypotéz a $Z$ počet zamietnutých hypotéz. 
Nás bude najviac zaujímať veľkosť $D$, pretože ide o~počet zamietnutých platných hypotéz. 

\begin{table}[h!]
    \centering
    \begin{tabular}{|c|c|c|c|}
      \hline
       & Platné $H_i$ & Neplatné $H_i$ & Celkom \\ \hline
      Zamietnuté $H_i$ & $D$ & $E$ & $Z$ \\ \hline
      Nezamietnuté $H_i$ & $F$ & $G$ & $K$-$Z$ \\ \hline
      Celkom & $K_0$ & $K$-$K_0$ & $K$ \\ \hline
    \end{tabular}
    \captionsetup{justification=centering}
    \caption{Počet hypotéz v každej možnosti, ktorá nastáva}
    \label{tab02:1}
\end{table}

Familywise Error Rate sa dá zapísať aj pomocou tohto značenia, platí 
$$ {\rm FWER} = P (D \geq 1). $$ 
Pri niektorých korekciách sa často kontroluje chyba ${\rm FDR}$, ktorá nie je taká striktná ako ${\rm FWER}$. 

\begin{definicia}\label{def6}
  False Discovery Rate definujeme ako strednú hodnotu z podielu zamietnutých platných hypotéz k zamietnutým hypotézam  
  za predpokladu zamietnutia aspoň jednej hypotézy. Budeme ju značiť ${\rm FDR}$. 
  Platí 
  $$ {\rm FDR} = E \left( \frac{D}{Z} ~ \bigg| ~ Z>0 \right) P(Z>0). $$
\end{definicia}

Ak sú všetky nulové hypotézy pravdivé, ${\rm FWER}$ a ${\rm FDR}$ sa rovnajú, dôkaz nájdeme v článku \cite{Benjamini&Hochberg95}. 
V ostatných prípadoch platí ${\rm FDR} \leq {\rm FWER}$. 
Pokiaľ kontrolujeme Familywise Error Rate, kontrolujeme zároveň aj False Discovery Rate. 
Opačná implikácia neplatí. 

%\chapter{Základné pojmy}

V tejto práci budeme predpokladať, že čitateľ pozná základné pojmy z matematickej štatistiky ako pravdepodobnostný a parametrický priestor, náhodný výber, 
model, testová štatistika a kritický obor.
V prvej kapitole zavedieme pojmy z matematickej štatistiky, ktoré pre nás budú kľúčové pri práci s mnohonásobným testovaním. 

V celej kapitole budeme predpokladať, že máme náhodný výber $${\bf X} = (X_1, X_2, \dots, X_n)^T$$ zložený z náhodných veličín, 
ktoré majú rozdelenie $F_X \in \mathcal{F}$, kde $\mathcal{F} = \{F(\theta),$ $\theta~\in~\Theta$ je neznámy parameter$\}$ je model. 
Teda $\Theta$ je množina všetkých hodnôt parametru v modele $\mathcal{F}$. 
Skutočný parameter budeme značiť $\theta_X$. 

Označme $\Theta_0$ a $\Theta_1$ dve disjunktné podmnožiny parametrického priestoru $\Theta$. 
Pri testovaní nás bude zaujímať konkrétna hodnota parametru $\theta_X$, avšak stačí zistiť, 
či $\theta_X \in \Theta_0$ alebo $\theta_X \in \Theta_1$. 

\begin{definicia}\label{def1}
  Množinu $\Theta_0$ nazývame nulová hypotéza a $\Theta_1$ alternatívna hypotéza. 
  Množinu všetkých rozdelení z modelu $\mathcal{F}$, ktorých parametre splňajú nulovú hypotézu, budeme značiť $\mathcal{F}_0$, 
  podobne množinu rozdelení s parametrami spĺňajúcimi alternatívnu hypotézu značíme $\mathcal{F}_1$. 
\end{definicia}  

Pokiaľ chceme testovať, či je parameter $\theta_X$ rovný konkrétnej hodnote $\theta_0$, môžeme zvoliť nulovú hypotézu 
$H_0: \theta_X = \theta_0$ a alternatívu $H_1: \theta \neq \theta_0$. Tento test budeme naývať obojstranný. 
Ďalšou možnosťou je zvoliť nulovú hypotézu pomocou nerovnosti, teda $H_0: \theta_X \leq \theta_0$. Alternatívna hypotéza 
potom bude mať tvar $H_1: \theta_X > \theta_0$. Rovnosti môžu byť v tomto prípade otočené, avšak krajný bod intervalu musí 
byť vždy zahrnutý v nulovej hypotéze. Tento test sa nazýva jednostranný. 

\begin{definicia}\label{def2}
  Nech $\mathcal{S}({\bf X})$ je testová štatistika a $\mathcal{C}$ je kritický obor. Pomocou nich vieme definovať štatistický test. 
  Ak $\mathcal{S}({\bf X}) \in \mathcal{C}$, zamietame nulovú hypotézu v prospech alternatívnej hypotézy.  
  Ak $\mathcal{S}({\bf X}) \notin \mathcal{C}$, nemôžeme zamietnuť nulovú hypotézu v prospech alternatívnej hypotézy. 
\end{definicia}

Štatistický test nemusí v každom prípade rozhodnúť správne. 
Preto nastávajú 4 rôzne situácie vzhľadom k platnosti nulovej hypotézy a vyhodnotenia testu, 
ktoré môžeme prehľadne vidieť v~tabuľke \ref{tab01:1}. 

\begin{table}[h!]
  \centering
  \begin{tabular}{|c|c|c|}
    \hline
     & Platí $H_0$ & Neplatí $H_0$ \\ \hline
    Zamietame $H_0$ & Chyba 1.druhu & OK \\ \hline
    Nezamietame $H_0$ & OK & Chyba 2.druhu \\ \hline
  \end{tabular}
  \caption{Všetky možnosti, ktoré nastávajú pri testovaní hypotéz}
  \label{tab01:1}
\end{table}

V prípade zamietnutia platnej hypotézy hovoríme o chybe 1. druhu. 
Nezamietnutie neplatnej hypotézy sa nazýva chyba 2. druhu. 
Chyba 1. druhu je závažnejšia, preto budeme kontrolovať jej veľkosť a budeme chcieť, aby bola čo najmenšia.

Na kontrolu chyby 1. druhu sa používa hladina testu. Niekedy budeme používať takisto výraz hladina významnosti. 

\begin{definicia}\label{def3}
  Nech $\alpha \in (0,1)$ je vopred dané číslo, nech máme test s testovou štatistikou $S({\bf X})$ a kritickým oborom $\mathcal{C}$. 
  Hladina testu je rovná $\alpha$, ak je splnená podmienka $$ \sup_{F \in \mathcal{F}_0} P_F (S({\bf X}) \in \mathcal{C}) = \alpha, $$
  kde $P_F$  označuje pravdepodobnosť za predpokladu rozdelenia F.  
  V niektorých prípadoch je hladina testu dosiahnutá len asymptoticky, teda pre $n \longrightarrow \infty$ musí platiť upravená podmienka 
  $$ \sup_{F \in \mathcal{F}_0} \lim_{n \rightarrow \infty} P_F (S({\bf X}) \in \mathcal{C}) = \alpha. $$
\end{definicia}  

Hladina testu je pravdepodobnosť, že zamietneme nulovú hypotézu, ktorá v skutočnosti platí. 
Je to teda veľkosť chyby 1. druhu. 
Pri testovaní vždy na začiatku určíme na akej hladine významnosti budeme dané hypotézy testovať. 
V tejto práci budeme väčšinou voliť hladinu testu $\alpha=0.05$. 

Funkciu $\beta_n(F) = P_F (S({\bf X}) \in \mathcal{C})$, kde $F \in \mathcal{F}$, nazývame silofunkcia testu. 
V prípade, že $F \in \mathcal{F}_1$, číslo $\beta_n(F)$ nazývame sila testu a je to pravdepodobnosť zamietnutia neplatnej hypotézy. 
Pri testovaní chceme, aby sila testu bola čo najvyššia. Platí, že chyba 2. druhu je rovná $1-\beta_n(F)$. 

Pri testovaní hypotéz nebudeme vždy postupovať ako sme opísali v predchádzajúcej časti kapitoly. 
Hypotézy vieme otestovať aj pomocou p-hodnoty, ktorú teraz definujeme. 

\begin{definicia}\label{def4}
  Nech $\mathcal{S}(\bf{X})$ je testová štatistika, $\mathcal{C}$ je kritický obor a nech $s$ je hodnota testovej štatistiky $\mathcal{S}(\bf{X})$ 
  spočítaná z napozorovaných dát, ktoré chceme testovať. 
  P-hodnotu definujeme ako 
  \begin{itemize} 
    \item $ p = sup_{F \in \mathcal{F}_0} P_F (S({\bf X}) \leq s) $, 
    ak $\mathcal{C} = \langle c_U, \infty)$ pre nejaké $c_U \in \R$,
    \item $ p = sup_{F \in \mathcal{F}_0} P_F (S({\bf X}) \geq s) $, 
    ak $\mathcal{C} = (-\infty, c_L \rangle$ pre nejaké $c_L \in \R$,
    \item $ p = 2 \min \{P_F (\mathcal{S}({\bf X}) \leq s), P_F (\mathcal{S}({\bf X}) \geq s)\} $, 
    ak $\mathcal{C} = (-\infty, c_L \rangle \cup \langle c_U, \infty)$  
    pre nejaké $c_L, c_U \in \R$, $c_L < c_U$ a zároveň je splnená podmienka 
    \newline $ \sup_{F \in \mathcal{F}_0} P_F (\mathcal{S}({\bf X}) \leq c_L) 
    = \sup_{F \in \mathcal{F}_0} P_F (\mathcal{S}({\bf X}) \geq c_U) = \frac{\alpha}{2}. $
  \end{itemize}
\end{definicia}  

P-hodnota sa niekedy nazýva dosiahnutá hladina testu. 
Podmienka na konci definície musí byť splnená, aby celková hladina testu bola rovná $\alpha$. 

Ako sme zmienili vyššie, pomocou p-hodnoty vieme rozhodnúť, či máme hypotézu zamietnuť alebo nie. 
Budeme k tomu potrebovať nasledujúce tvrdenie, ktorého dôkaz viď \cite[Omelka, Tvrdenie 4.1]{Omelka18}. 

\begin{tvrd}\label{tvrd01}
  Nech $\mathcal{S}({\bf X})$ je testová štatistika so spojitým rozdelením. 
  Uvažujme test hypotézy $H_0$ proti alternatíve $H_1$ daný pravidlom 
  \begin{center}
    $H_0$ zamietame $\Longleftrightarrow$ $p \leq \alpha$, \\
    $H_0$ nezamietame $\Longleftrightarrow$ $p > \alpha$.\\
  \end{center}  
  Potom má tento test hladinu $\alpha$. 
\end{tvrd}  

%\chapter{Mnohonásobné testovanie}

V tejto kapitole budeme vysvetľovať ako funguje mnohonásobné testovanie, aké problémy nastávajú pri jeho používaní 
a definujeme chyby, ktoré budeme kontrolovať. 
V poslednej podkapitole ukážeme problém s hladinou testu pomocou obrázkov. 

\section{Úvod}

V tejto podkapitole budeme predpokladať, že máme $K$ hypotéz, ktoré chceme testovať, 
nulové hypotézy budeme značiť $H^{(i)}_0$, alternatívy $H^{(i)}_1$ 
s testovou štatistikou $S^{(i)}$ a kritickým oborom $C^{(i)}$, $i \in \{1, \dots, K\}$. 
Všetky nulové hypotézy budeme testovať simultánne a nezávisle na sebe, 
tento typ testovania sa nazýva mnohonásobné testovanie. 

Ako sme spomínali v prvej kapitole, pri testovaní hypotéz je potrebné kontrolovať hladinu testu, ktorú určíme na začiatku. 
Pre mnohonásobné testovanie zvolíme hladinu testu $\alpha$ a každý test $T^{(i)}$ budeme testovať na tejto hladine. 
To znamená, že pravdepodobnosť zamietnutia platnej nulovej hypotézy v teste $T^{(i)}$ bude rovná $\alpha$ pre každé $i \in \{1, \dots, K\}$.
$$ P_{H^{(i)}_0} (S^{(i)} \in C^{(i)}) = \alpha $$ 
Takisto vieme vyjadriť pravdepodobnosť nezamietnutia platnej hypotézy, teda pravdepodobnosť, že nespravíme chybu 1.druhu. 
$$ P_{H^{(i)}_0} (S^{(i)} \notin C^{(i)}) = 1 - \alpha $$ 
Pravdepodobnosť všetkých zamietnutí platných hypotéz je pravdepodobnosť zjednotenia všetkých zamietnutí hypotéz 
za predpokladu platnosti všetkých nulových hypotéz. 
Túto pravdepodobnosť budeme označovať $\alpha_K$ a formálne to môžeme zapísať nasledovným spôsobom. 
Skutočnosť, že všetky nulové hypotézy sú platné, budeme značiť $ \mathcal{H}_0 = {\bigcap_{i=1}^{K} H^{(i)}_0} $.
$$ P_{\mathcal{H}_0} ( \bigcup_{i=1}^{K} [S^{(i)} \in C^{(i)}] ) = \alpha_K $$ 
Je zrejmé, že $\alpha_K$ je väčšie ako $\alpha$, ktoré sme zvolili na začiatku. 
Túto skutočnosť neskôr ukážeme pomocou obrázkov. 

\section{Chyby}

Problém mnohonásobného testovania sa dá ukázať aj iným spôsobom. 
Podobne ako sme opísali v predchádzajúcej podkapitole, môžeme vyjadriť pravdepodobnosť, 
že nespravíme chybu 1. druhu ani pri jednom testovaní hypotézy $H^i$, kde $i \in \{1, \dots, K\}$. 
$$ P_{\mathcal{H}_0} ( \bigcup_{i=1}^{K} [S^{(i)} \notin C^{(i)}] ) = (1 - \alpha)^K $$
Pravdepodobnosť zamietnutia aspoň jednej platnej hypotézy zapíšeme nasledujúcim spôsobom. 
$$ P_{\mathcal{H}_0} ( \exists i \in \{1, \dots, K\}: [S^{(i)} \in C^{(i)}] ) 
   = 1 - P_{\mathcal{H}_0} ( \bigcup_{i=1}^{K} [S^{(i)} \notin C^{(i)}] ) = 1 - (1 - \alpha)^K $$    

\begin{definicia}\label{def5} 
  Predpokladajme, že máme $K$ testov, kde $S^{(i)}$ sú ich testové štatistiky a $C^{(i)}$ kritické obory, $i \in \{1, \dots, K\}$. 
  Pravdepodobnosť zamietnutia aspoň jednej platnej hypotézy sa nazýva familywise error rate, 
  budeme ju značiť ${\rm FWER}$. Teda platí 
  $$ {\rm FWER} = P_{\mathcal{H}_0} ( \exists i \in \{1, \dots, K\}: [S^{(i)} \in C^{(i)}] ). $$
\end{definicia}

Pri mnohonásobnom testovaní budeme kontrolovať hlavne familywise error rate a budeme chcieť, 
aby táto chyba bola čo najmenšia. 

Problém s mnohonásobným testovaním ukážeme na príklade s konkrétnymi číslami. 
Nech $\alpha = 0.05$, počet hypotéz $K = 10$. Jednotlivé hypotézy budeme testovať na hladine $\alpha$. 
Pravdepodobnosť, že spravíme minimálne jednu chybu pri mnohonásobnom testovaní, 
bude ${\rm FWER} = 1 - (1 - 0.05)^{10} = 0.4012631$. 

\begin{figure}[h!]
  \centering
  \includegraphics[width=\linewidth]{C:/Users/KIKA/Desktop/BP/R/obr1.pdf}
  \caption{Familywise error rate pri rôznych počtoch hypotéz s rôznymi hodnotami $\alpha$, 
  če}
  \label{obr02:1}
\end{figure}

Na Obrázku \ref{obr02:1} vidíme veľkosť chyby ${\rm FWER}$ podľa počtu hypotéz, 
ktoré testujeme mnohonásobným testovaním s hladinou významnosti $\alpha = 0.05$. 
Pravdepodobnosť, že spravíme aspoň jednu chybu je vyššia ako $0.5$ už pri testovaní $15$ hypotéz. 

Aby bola táto chyba menšia alebo rovná ako $\alpha$, ktoré sme zvolili na začiatku, 
je nutné testovať jednotlivé hypotézy s menšou hladinou významnosti. 

V Tabuľke \ref{tab02:1} môžeme vidieť všetky možnosti, ktoré môžu nastať pri mnohonásobnom testovaní, 
pričom $D$, $E$, $F$, $G$ označujú počty hypotéz $H^{(i)}_0$ v každej možnosti pre $i \in \{1, \dots, K\}$, kde $K$ je ich súčet. 
$K_0$ označuje počet platných hypotéz a $Z$ počet zamietnutých hypotéz. 
Nás bude najviac zaujímať veľkosť $D$, pretože je to počet zamietnutých platných hypotéz. 

\begin{table}[h!]
    \centering
    \begin{tabular}{|c|c|c|c|}
      \hline
       & Platné $H^{(i)}_0$ & Neplatné $H^{(i)}_0$ & Celkom \\ \hline
      Zamietnuté $H^{(i)}_0$ & $D$ & $E$ & $Z$ \\ \hline
      Nezamietnuté $H^{(i)}_0$ & $F$ & $G$ & $K$-$Z$ \\ \hline
      Celkom & $K_0$ & $K$-$K_0$ & $K$ \\ \hline
    \end{tabular}
    \caption{Počet hypotéz v každej možnosti, ktorá nastáva}
    \label{tab02:1}
\end{table}

Familywise error rate sa dá zapísať aj pomocou tohto značenia, platí 
${\rm FWER} = P (D \geq 1)$. 
Pri niektorých metódach sa často kontroluje chyba, ktorá nie je taká striktná ako ${\rm FWER}$. 

\begin{definicia}\label{def6}
  False discovery rate definujeme ako strednú hodnotu z podielu zamietnutých platných hypotéz k zamietnutým hypotézam  
  za predpokladu zamietnutia aspoň jednej hypotézy. Budeme ju značiť ${\rm FDR}$. 
  Platí 
  $$ {\rm FDR} = E \left( \frac{D}{Z} \Big| Z>0 \right) P(Z>0). $$
\end{definicia}

Ak sú všetky nulové hypotézy pravdivé, ${\rm FWER}$ a ${\rm FDR}$ sa rovnajú \cite{Benjamini&Hochberg95}. 
V opačnom prípade platí ${\rm FDR} \leq {\rm FWER}$. 
Teda pokiaľ kontrolujeme familywise error rate, kontrolujeme zároveň aj false discovery rate. 

Chyba ${\rm FWER}$ sa dá kontrolovať dvomi rôznymi spôsobmi, 
pri prvom predpokladáme platnosť všetkých nulových hypotéz, ktoré testujeme, 
ako sme uviedli v definícii \ref{def5}. 
Nazýva sa to slabá kontrola chyby ${\rm FWER}$. 
V tomto prípade máme zaručenú ${\rm FWER}$ na hladine $\alpha$ len za platnosti všetkých hypotéz, 
teda väčšinou nie je vhodné ich využívať. 

Označme $\mathcal{H} = \{ H_0^{(1)}, \dots, H_0^{(K)} \}$ množinu všetkých hypotéz, ktoré chceme testovať. 
Nech $\mathcal{H}'$ je podmnožina $\mathcal{H}$. 
Skutočnosť, že platia všetky hypotézy v $\mathcal{H}$, respektíve v $\mathcal{H}'$, 
označíme ${\mathcal{H}}_0$, respektíve ${\mathcal{H}}'_0$. 
Silná kontrola chyby ${\rm FWER}$ sa dá zapísať ako 
$$ P_{{\mathcal{H}}'_0} \left( \exists H_0^{(i)} \in {\mathcal{H}}'_0: [S^{(i)} \in C^{(i)}] \right) \leq \alpha, $$
pre každú $\mathcal{H}' \subseteq \mathcal{H}$. 
Je to pravdepodobnosť zamietnutia aspoň jednej platnej hypotézy za predpokladu, 
že je platná ktorákoľvek podmnožina hypotéz. 

\section{Problém mnohonásobného testovania}

V tejto podkapitole ukážeme problém mnohonásobného testovania pomocou obrázkov. 
Na začiatku ukážeme aká je skutočná hladina významnosti v prípade testovanie jednej hypotézy. 
Hypotézy budeme testovať na hladine $\alpha = 0.05$. 

V celej podkapitole budeme predpokladať, že máme náhodný výber
${\bf X} = (X_1, X_2, \dots, X_n)^T$ z rozdelenia $N(\mu, \sigma^2)$, 
teda budeme pracovať s modelom $\mathcal{F} = \{ N(\mu, \sigma^2),~\mu \in \R,~\sigma^2>0 \}$. 

Najskôr budeme testovať hypotézu $H_0$ proti alternatíve $H_1$, ktoré budú tvaru 
$$ H_0: \mu = 0,~H_1: \mu \neq 0. $$
Jednovýberovým testom budeme testovať vygenerované dáta z normovaného normálneho rozdelenia. 
Rozsah dát bude $100$. 

Tento postup budeme opakovať niekoľko krát po sebe, pričom počet testovaní budeme meniť, 
postupne $20, 200, 2000, 20000$. 
Potrebujeme zistiť počet zamietnutých platných hypotéz. 
Dáta sme vygenerovali so strednou hodnotou rovnou $0$, 
všetky hypotézy sú platné a podiel zamietnutých hypotéz k počtu testovaní bude rovný skutočnej hladine testu. 

\begin{figure}[h!]
  \centering
  \includegraphics[width=\linewidth]{C:/Users/KIKA/Desktop/BP/R/obr2.pdf}
  \caption{Hladina jednovýberového testu pre rôzne počty opakovaní}
  \label{obr02:2}
\end{figure}

Na Obrázku \ref{obr02:2} sa skutočná hladina testu pohybuje v blízkosti zvolenej hladiny $\alpha = 0.05$.             

V tomto prípade budeme zároveň testovať dve hypotézy, jedna bude testovať strednú hodnotu a druhá rozptyl náhodného výberu 
a budú tvaru
$$ H_0^{(1)}: \mu = 0,~H_1^{(1)}: \mu \neq 0, $$
$$ H_0^{(2)}: \sigma^2 = 1,~H_1^{(2)}: \sigma^2 \neq 1. $$ 

Každú nulovú hypotézu otestujeme použitím jednovýberového testu. 
Mnohonásobné testovanie zopakujeme niekoľko krát po sebe, počet testovaní budeme meniť, 
postupne $20, 200, 2000, 20000$. 
Dáta s rozsahom $100$ boli vygenerované so strednou hodnotou rovnou $0$ a smerodajnou odchýlkou rovnou $1$,  
teda všetky náhodné výbery majú rovnakú strednú hodnotu a rozptyl, hypotézy sú platné. 
Podiel zamietnutých hypotéz k počtu testovaní je skutočná hladina testu. 

\begin{figure}[h!]
  \centering
  \includegraphics[width=\linewidth]{C:/Users/KIKA/Desktop/BP/R/obr3.pdf}
  \caption{Hladina mnohonásobného testovania po $1000$ opakovaniach pre rôzne počty pozorovaní}
  \label{obr02:3}
\end{figure}

Na Obrázku \ref{obr02:3} je skutočná hladina testu skoro dvojnásobne vyššia 
ako pôvodne zvolená hladina $\alpha = 0.05$. 

\chapter{Korekcie}

V~predchádzajúcej kapitole sme opísali problém, ktorý nastáva pri~mnohonásobnom testovaní. 
Problém s~hladinou významnosti sa dá vyriešiť použitím rôznych korekcií, 
ktoré korigujú celkovú hladinu významnosti jednotlivých testov.  

Nech $p_1, \dots, p_K$ sú postupne p-hodnoty testov nulových hypotéz $H_1, \dots, H_K$. 
P-hodnoty zoradíme podľa veľkosti a~následne označíme v~poradí
$$ p_{(1)} \leq p_{(2)} \leq \dots \leq p_{(K)}. $$ 
Nulové hypotézy patriace k~zoradeným p-hodnotám označíme 
$$ H_{(1)}, H_{(2)}, \dots, H_{(K)}. $$ 
Tento zápis budeme potrebovať na~definovanie niektorých korekcií. 

Korekcie, ktoré upravujú hladinu testov delíme na~dve~skupiny: 
\begin{itemize}
  \item simultánne zamietajúce, hypotézy sa zamietajú nezávisle na~sebe, 
  \item postupne zamietajúce, hypotézy zamietame pomocou zoradených p-hodnôt. 
\end{itemize}

Korekcie rozlišujeme taktiež podľa toho, ktorú chybu kontrolujú. 
Väčšina korekcií kontroluje silnejšiu verziu chyby ${\rm FWER}$, ktorá zároveň kontroluje aj~chybu ${\rm FDR}$, 
ako sme uviedli v~predchádzajúcej kapitole.  
Existujú však aj korekcie kontrolujúce len chybu ${\rm FDR}$, 
teda nie sú až~také striktné. 

Väčšina korekcií upravujúcich hladinu testu je založená na~Booleovej nerovnosti, 
ktorú uvedieme aj~s~dôkazom. 

\begin{tvrd}\label{tvrd02}
  Nech $B_1, \dots, B_n$ sú náhodné javy, $n \in \N$. 
  Potom platí 
  $$ P \left( \bigcup_{i=1}^{n} B_i \right) \leq \sum_{i=1}^{n} P \left( B_i \right). $$
\end{tvrd}    
\begin{dokaz}
  Nerovnosť dokážeme indukciou. 
  Pre $n = 1$ nerovnosť platí. 
  Budeme predpokladať, že nerovnosť platí pre~$n$, teda
  $$ P \left( \bigcup_{i=1}^{n} B_i \right) \leq \sum_{i=1}^{n} P \left( B_i \right). $$ 
  Dokážeme, že nerovnosť platí aj~pre~$n+1$, 
  použijeme k~tomu rovnosť
  $$ P \left( \bigcup_{i=1}^{n+1} B_i \right) 
    = P \left( \bigcup_{i=1}^{n} B_i \right) + P ( B_{n+1} ) - P \left( \bigcup_{i=1}^{n} B_i \cap B_{n+1} \right). $$
  Keďže $$ P \left( \bigcup_{i=1}^{n} B_i \cap B_{n+1} \right) \geq 0, $$
  platí 
  $$ P \left( \bigcup_{i=1}^{n+1} B_i \right) 
    \leq P \left( \bigcup_{i=1}^{n} B_i \right) + P ( B_{n+1} ) 
    \leq \sum_{i=1}^{n} P \left( B_i \right) + P ( B_{n+1} ) = \sum_{i=1}^{n+1} P \left( B_i \right). $$ 
  Druhá nerovnosť vyplýva z~indukčného predpokladu. 
\end{dokaz}

\section{Bonferroni}

Najznámejšou korekciou pri~úpravách mnohonásobného testovania je Bonferroniho korekcia. 
Je založená na~Booleovej nerovnosti, ktorú sme ukázali na~začiatku kapitoly. 

Podľa Tvrdenia \ref{tvrd01} zamietame nulovú hypotézu práve vtedy, keď je p-hodnota menšia alebo rovná ako~hladina testu. 
Toto tvrdenie použijeme aj~pri~odvodení tejto korekcie. 
Nech $p_1, \dots, p_K$ sú postupne p-hodnoty hypotéz $H_1, \dots, H_K$. 
Bonferroniho korekcia zamieta nulovú hypotézu práve vtedy, keď $p_i \leq \frac{\alpha}{K}$ 
pre~každé $i \in \{1, \dots, K\}$. 
Chceme, aby veľkosť ${\rm FWER}$ bola menšia alebo rovná ako hladina testu $\alpha$, ktorú určíme na~začiatku. 
Nech $\mathcal{H'}_0$ je množina platných hypotéz, ich počet je~$K'$. 
Nech $I$ je~indexová množina platných hypotéz, teda $I~=~\{ i~\in~\{1, \dots, K\}; H_i ~\in~{\mathcal{H}}'_0 \}$. 
Potom platí 
\begin{center}
  $$ {\rm FWER} =  P_{\mathcal{H}'_0} \left( \exists H_i \in {\mathcal{H}}'_0: \left[ \mathcal{S}^{(i)} \in \mathcal{C}^{(i)} \right] \right) 
  = P_{\mathcal{H'}_0} \left( \bigcup_{i \in I} \left[ \mathcal{S}^{(i)} \in \mathcal{C}^{(i)} \right] \right) $$
  $$ \leq \sum_{i \in I} P_{H^i} \left( \mathcal{S}^{(i)} \in \mathcal{C}^{(i)} \right) 
  = K' \alpha_0 \leq K \alpha_0 = K \frac{\alpha}{K}  = \alpha, $$
\end{center}  
kde $\alpha_0 = \frac{\alpha}{K}$ je hladina jednotlivých testov. 

Už~pri~testovaní $10$ hypotéz by sme museli použiť na~jednotlivé testy hladinu $\alpha_0 = 0.005$. 
To~je~veľmi nízka hladina testu a~niektoré neplatné hypotézy by nemuseli byť zamietnuté, 
teda sila testu je taktiež nižšia. 
Z odvodenia korekcie vidíme, že kontroluje chybu ${\rm FWER}$ v~silnejšom zmysle. 

\section{Šidák}

Šidákova korekcia kontroluje taktiež chybu ${\rm FWER}$ v~silnejšom zmysle. 
Patrí medzi korekcie, ktoré testujú hypotézy simultánne. 
Jej~odvodenie vyplýva z~alternatívneho zápisu ${\rm FWER}$, 
ktorý sme uviedli pred~Definíciou~\ref{def5}, 
$$ {\rm FWER} = 1 - ( 1 - \alpha_0)^K, $$ 
kde~$K$ je počet všetkých hypotéz a~$\alpha_0$ hladina jednotlivých testov. 
Keď rovnosť prepíšeme iným spôsobom, dostaneme
$$ \alpha_0 = 1 - ( 1 - {\rm FWER} )^{\frac{1}{K}}. $$ 
Ak~chceme, aby ${\rm FWER}$ bola menšia alebo rovná ako $\alpha$, zvolíme 
$$ \alpha_0 = 1 - ( 1 - \alpha )^{\frac{1}{K}}. $$
Podrobné odvodenie korekcie a~jej~dôkaz nájdeme v článku \cite{Sidak67}. 

\begin{table}[h!]
  \centering
  \begin{tabular}{c|r@{.}lr@{.}lr@{.}lr@{.}lr@{.}lr@{.}l}
     & \multicolumn{2}{c}{{\bf BONF}} & \multicolumn{2}{c}{{\bf ŠIDÁK}} 
     & \multicolumn{2}{c}{{\bf BONF}} & \multicolumn{2}{c}{{\bf ŠIDÁK}} 
     & \multicolumn{2}{c}{{\bf BONF}} & \multicolumn{2}{c}{{\bf ŠIDÁK}} \\ 
     {\bf $\alpha$} & \multicolumn{4}{c}{{\bf $0.01$}} & \multicolumn{4}{c}{{\bf $0.05$}} & \multicolumn{4}{c}{{\bf $0.1$}} \\ \hline
    {\bf 1} & 0&0100 & 0&0100 & 0&0500 & 0&0500 & 0&1000 & 0&1000 \\
    {\bf 2} & 0&0050 & 0&0050 & 0&0250 & 0&0253 & 0&0500 & 0&0513 \\
    {\bf 3} & 0&0033 & 0&0033 & 0&0167 & 0&0169 & 0&0333 & 0&0345 \\
    {\bf 4} & 0&0025 & 0&0025 & 0&0125 & 0&0127 & 0&0250 & 0&0260 \\
    {\bf 5} & 0&0020 & 0&0020 & 0&0100 & 0&0102 & 0&0200 & 0&0209 \\
    {\bf 6} & 0&0017 & 0&0017 & 0&0083 & 0&0085 & 0&0167 & 0&0174 \\
    {\bf 7} & 0&0014 & 0&0014 & 0&0071 & 0&0073 & 0&0143 & 0&0149 \\ 
    {\bf 8} & 0&0013 & 0&0013 & 0&0063 & 0&0064 & 0&0125 & 0&0131 \\ 
    {\bf 9} & 0&0011 & 0&0011 & 0&0056 & 0&0057 & 0&0111 & 0&0116 \\ 
    {\bf 10} & 0&0010 & 0&0010 & 0&0050 & 0&0051 & 0&0100 & 0&0105 \\ 
  \end{tabular}
  \caption{Veľkosti $\alpha_0$ pri~použití Bonferroniho a~Šidákovej korekcie 
  pre~rôzne zvolené hladiny $\alpha$}
  \captionsetup{justification=centering}
  \label{tab03:1}
\end{table}

Z~Tabuľky~\ref{tab03:1} vidíme, že~Šidákova korekcia má v~porovnaní s~Bonferroniho korekciou 
väčšie hladiny jednotlivých testov, to~znamená, že aj~sila testu bude vyššia. 
Pri~vyšších počtoch hypotéz sú upravené hladiny týchto dvoch korekcií podobné. 
Pre~$\alpha = 0.01$ nie je vzhľadom k~zaokrúhleniu vidieť rozdiel medzi korekciami. 

\section{Holm}

Holmova korekcia je opäť založená na~Booleovej nerovnosti. 
Patrí medzi korekcie s~postupným zamietaním hypotéz. 

Na~definovanie korekcie budeme potrebovať zoradené p-hodnoty $p_{(i)}$ a k~nim príslušné hypotézy $H_{(i)}$, 
ako sme zaviedli na~začiatku kapitoly. 
Odvodenie korekcie nájdeme v článku~\cite{Holm79}. 
\newline ${\bf Procedúra:}$ 
\begin{itemize}
  \item Ak $p_{(1)} \leq \frac{\alpha}{K}$, zamietame hypotézu  $H_{(1)}$ a~pokračujeme ďalším krokom. 
  V~opačnom prípade nezamietame hypotézy $H_{(1)}, H_{(2)}, \dots, H_{(K)}$. 
  Zo~zoradených p-hodnôt vyplýva  $\frac{\alpha}{K} < p_{(1)} \leq p_{(2)} \leq \dots \leq p_{(K)}$, 
  podľa Bonferroniho korekcie nezamietame žiadnu hypotézu. 
  \item Ak $p_{(i)} \leq \frac{\alpha}{K-i+1}$, zamietame hypotézu $H_{(i)}$ a~pokračujeme ďalším krokom. 
  V~opačnom prípade nezamietame hypotézy $H_{(i)}, H_{(i+1)}, \dots, H_{(K)}$. 
  \item Ak $p_{(K)} \leq \alpha$, zamietame $H_{(K)}$ a~procedúra končí. 
  V~opačnom prípade nezamietame hypotézu $H_{(K)}$. 
\end{itemize}  

\begin{tvrd}\label{tvrd03}
  Holmova korekcia kontroluje, aby chyba ${\rm FWER}$ bola menšia alebo rovná ako $\alpha$. 
\end{tvrd} 
\begin{dokaz}
  Nech $\mathcal{H}'_0 \subseteq \mathcal{H}$ je~množina pravdivých hypotéz, 
  nech~$I$ je jej~množina indexov, $K'$~je počet prvkov množiny~$I$. 
  Z~Booleovej nerovnosti vyplýva nasledujúca nerovnosť  
  \begin{center}
  $$ {\rm FWER} = P_{\mathcal{H}'_0} \left( \exists i \in I : p_i \leq \frac{\alpha}{K'} \right) 
    = P_{\mathcal{H}'_0} \left( \bigcup_{i \in I} \left[ p_i \leq \frac{\alpha}{K'} \right] \right)  $$
  $$ \leq \sum_{i \in I}  P_{H_i} \left( p_i \leq \frac{\alpha}{K'} \right) 
    = K' \frac{\alpha}{K'} = \alpha. $$
  \end{center}
  Pokiaľ platí 
  $$ \forall i \in I : p_i > \frac{\alpha}{K'}, $$ 
  existuje $K'$ hypotéz, ktoré sú väčšie ako $\frac{\alpha}{K'}$ a~určite platí
  $$ p_{(K)} > \dots > p_{(K-K'+1)} > \frac{\alpha}{K'}. $$ 
  Procedúra sa zastaví najneskôr v~kroku $K-K'+1$. 
  To~implikuje, že množina všetkých hypotéz s~p-hodnotou menšou ako $\frac{\alpha}{K'}$ 
  obsahuje aj~množinu platných hypotéz. 
\end{dokaz}  

Z~dôkazu Tvrdenia~\ref{tvrd03} vyplýva, že Holmova korekcia kontroluje chybu ${\rm FWER}$ v~silnejšom zmysle. 

\section{Simes}

Na~definíciu Simesovej korekcie budeme opäť používať zápis zoradených p-hodnôt a~k~nim príslušných hypotéz. 
Túto korekciu zaraďujeme medzi simultánne zamietajúce. 
\newline ${\bf Procedúra:}$
\begin{itemize}
  \item Ak $p_{(i)} \leq \frac{i\alpha}{K}$ pre~každé $i \in \{1, \dots, K\}$, zamietame ${\mathcal{H}}$, 
  kde ${\mathcal{H}}$ označuje množinu všetkých nulových hypotéz. 
\end{itemize}  

\begin{tvrd}\label{tvrd04}
  Nech $p_{(1)}, \dots, p_{(K)}$ sú zoradené štatistiky z~$K$~nezávislých náhodných veličín 
  z~rovnomerného rozdelenia na~$(0,1)$ a~nech $\alpha \in (0,1)$. 
  Potom platí
  $$ P \left( p_{(i)} > \frac{i\alpha}{K} ;~i \in \{1, \dots, K\} \right) = 1 - \alpha. $$
\end{tvrd}  

Dôkaz Tvrdenia~\ref{tvrd04} sa~nachádza v~článku \cite{Simes86}. 
Táto korekcia kontroluje ${\rm FWER}$ len~v~slabšom zmysle, 
teda zaručuje ${\rm FWER} \leq \alpha$ iba~v~prípade, že platia všetky nulové hypotézy. 

\section{Hochberg}

V~článku~\cite{Hochberg88} je porovnaná Holmova a~Simesova korekcia. 
Keďže Simesova korekcia kontroluje chybu ${\rm FWER}$ len v~slabšom zmysle, 
Hochberg vytvoril korekciu odvodenú z~týchto dvoch korekcií, 
ktorá bude kontrolovať silnú verziu chyby ${\rm FWER}$. 
Táto korekcia má~podobný zápis ako Holmova korekcia, 
avšak procedúra začína testovaním hypotézy s~najväčšou p-hodnotou. 
\newline ${\bf Procedúra:}$
\begin{itemize}
  \item Ak $p_{(K)} \leq \alpha$, zamietame hypotézy $H_{(K)}, H_{(K-1)}, \dots, H_{(1)}$. 
  V~opačnom prípade nezamietame hypotézu $H_{(K)}$ a~pokračujeme ďalším krokom. 
  \item Ak $p_{(i)} \leq \frac{\alpha}{K-i+1}$, zamietame hypotézy $H_{(K-i+1)}, H_{(K-i)}, \dots, H_{(1)}$,
  V~opačnom prípade nezamietame hypotézu $H_{(K-i+1)}$ a~pokračujeme ďalším krokom. 
  \item Ak $p_{(1)} \leq \frac{\alpha}{K}$, zamietame $H_{(1)}$. 
  V~opačnom prípade nezamietame hypotézu~$H_{(1)}$. 
\end{itemize}  

\begin{tvrd}\label{tvrd05}
  Nech $j \in \{1, \dots, K\}$. Ak pre~každé $i \in \{1, \dots, j\}$ platí $p_{(i)} \leq \frac{\alpha}{K-i+1}$, 
  tak~Simesova korekcia zamieta všetky $H_{(i)}$ pre~$i \leq j$. 
\end{tvrd}  
\begin{dokaz}
  Podľa predpokladu pre~každé $i \in \{1, \dots, j\}$ platí $p_{(i)} \leq \frac{\alpha}{K-i+1}$. 
  Označme $\mathcal{H}' \subseteq \mathcal{H}$ množinu hypotéz, ktoré spĺňajú túto nerovnosť, 
  $\mathcal{H}' = \{ H_{(1)}, \dots, H_{(j)} \}$. 
  Pre~každé $i \leq j$ platí nerovnosť $\frac{\alpha}{K-i+1} \leq \frac{i \alpha}{j}$, potom 
  $$ p_{(i)} \leq \frac{\alpha}{K-i+1} \leq \frac{i \alpha}{j}. $$ 
  Je~splnená podmienka Simesovej korekcie a zamietame hypotézy $H_{(1)}, \dots, H_{(j)}$. 
\end{dokaz}  

\section{Benjamini-Hochberg}

Podľa~článku~\cite{Benjamini&Hochberg95} nie je vždy potrebné kontrolovať chybu ${\rm FWER}$, 
pretože sa pozeráme len na~to, či~nastala chyba alebo nie. 
Ich~korekcia bude kontrolovať chybu ${\rm FDR}$, 
konkrétne strednú hodnotu podielu neplatných zamietnutých hypotéz k~všetkým zamietnutým hypotézam. 

Táto korekcia nie je výhodnejšia, pokiaľ sú všetky nulové hypotézy pravdivé, 
pretože v~tomto prípade sa~chyby ${\rm FWER}$ a~${\rm FDR}$ rovnajú. 

Budeme pracovať so~zoradenými p-hodnotami ako pri~predchádzajúcich korekciách. 
\newline${\bf Procedúra:}$
\begin{itemize}
  \item Nech $j$ je najväčšie~$i$, pre~ktoré platí $p_{(i)} \leq \frac{i\alpha}{K}$.  
  Potom zamietame všetky $H_(i)$ pre~$i \in \{1, \dots, j\}$. 
\end{itemize}  

\begin{tvrd}\label{tvrd06}
  Pre nezávislé testové štatistiky a pre~akékoľvek zostavenie nepravdivých nulových hypotéz
  kontroluje Benjamini-Hochbergova korekcia chybu ${\rm FDR}$ na~hladine $\alpha$. 
\end{tvrd} 
\begin{dokaz}
  Máme $K$~nezávislých p-hodnôt, ktoré patria k~nulovým hypotézam. 
  Nech $K_0$~je počet p-hodnôt príslušných k~pravdivým hypotézam a~$K_1 = K - K_0$~počet p-hodnôt príslušných k~nepravdivým hypotézam. 
  P-hodnoty príslušné k~platným hypotézam označíme $p^{(1)}, \dots, p^{(K_0)}$ a 
  p-hodnoty príslušné k~neplatným hypotézam $p^{(K_0+1)}, \dots, p^{(K)}$. 
  Použijeme značenie ako v~Tabuľke~\ref{tab02:1}, pripomeňme, že ${\rm FDR} = E \left( \frac{D}{Z} \right)$. 
  Benjamini-Hochbergerova korekcia spĺňa 
  $$ E \left( \frac{D}{Z} ~ \bigg| ~ p_{K_0+1} = p^{(1)}, \dots, p_{K} = p^{(K_1)} \right) \leq \frac{K_0}{K} \alpha \leq \alpha. $$
  Tvrdenie vyplýva z~nerovnosti, jej~dôkaz nájdeme v~článku~\cite{Benjamini&Hochberg95}. 
\end{dokaz}  


\chapter{Porovnanie korekcií}

V~tejto kapitole porovnáme korekcie, ktoré sme opísali v~predchádzajúcej kapitole. 
Korekcie porovnáme z~dvoch rôznych hladísk. 
V~prvej podkapitole sa~budeme pozerať na~hladinu významnosti jednotlivých korekcií. 
Neskôr porovnáme korekcie vzhľadom k~ich~sile. 
Na konci kapitoly ukážeme, že korekcie nemajú vplyv na konzistenciu testu. 

\section{Z~hľadiska hladiny testu}

Predpokladajme, že máme $K$~náhodných výberov s rozsahom $n=100$. 
\begin{center} 
    $ {\bf X}_1 = (X_{11}, X_{12}, \dots, X_{1n}) $ \\
    $ {\bf X}_2 = (X_{21}, X_{22}, \dots, X_{1n}) $ \\
    $\vdots$ \\
    $ {\bf X}_K = (X_{K1}, X_{K2}, \dots, X_{Kn}) $ \\
\end{center}
Strednú hodnotu náhodného výberu ${\bf X}_i$ označíme $\mu_i$, $i \in \{ 1, \dots, K\}$. 
Budeme testovať $K$~hypotéz v~rovnakom čase, ktoré budú tvaru 
\begin{center} 
$ H_1: \mu_1 = 0, $ \\
$ \vdots $ \\
$ H_K: \mu_K = 0. $ \\
\end{center}
Aby sme zistili hladinu významnosti mnohonásobného testovania, 
dáta vygenerujeme z~normovaného normálneho rozdelenia. 
Teda všetky hypotézy sú platné. 
Mnohonásobné testovanie prevedieme opakovane $100000$-krát na rôzne vygenerovaných dátach, 
pričom chceme, aby bola dodržaná hladina významnosti $\alpha=0.05$. 
Pri každom opakovaní budeme zisťovať koľko hypotéz bolo zamietnutých, 
pričom počet hypotéz budeme meniť. 
Následne spočítame koľkokrát z $100000$ opakovaní sme zamietli určité počty hypotéz. 
V práci uvedieme podiel tohto počtu opakovaní k všetkým opakovaniam. 
V~prvom prípade budeme testovať $3$~hypotézy. 

\begin{table}[h!]
  \centering
  \begin{tabular}{c|r@{.}lr@{.}lr@{.}lr@{.}lr@{.}lr@{.}lr@{.}l}
     & \multicolumn{2}{c}{\bf BONF} & \multicolumn{2}{c}{\bf ŠIDÁK} & \multicolumn{2}{c}{\bf HOLM} & \multicolumn{2}{c}{\bf SIMES} 
     & \multicolumn{2}{c}{\bf HOCH} & \multicolumn{2}{c}{\bf BH} & \multicolumn{2}{c}{\bf -} \\ \hline
    {\bf 1} & 0&0475 & 0&0483 & 0&0468 & 0&0000 & 0&0469 & 0&0458 & 0&1360 \\ 
    {\bf 2} & 0&0009 & 0&0009 & 0&0017 & 0&0000 & 0&0015 & 0&0034 & 0&0073 \\ 
    {\bf 3} & 0&0000 & 0&0000 & 0&0001 & 0&0493 & 0&0000 & 0&0001 & 0&0001 \\ \hline
    {\bf $\sum$} & 0&0484 & 0&0492 & 0&0486 & 0&0493 & 0&0484 & 0&0493 & 0&1434 \\ 
  \end{tabular}
  \caption{Podiel počtu zamietnutých hypotéz v~každom opakovaní k počtu opakovaní, 
  pri~testovaní $3$~hypotéz rôznymi korekciami s~počtom opakovaní $100000$}
  \captionsetup{justification=centering}
  \label{tab04:1}
\end{table}

\begin{table}[h!]
  \centering
  \begin{tabular}{c|r@{.}lr@{.}lr@{.}lr@{.}lr@{.}lr@{.}lr@{.}l}
    & \multicolumn{2}{c}{\bf BONF} & \multicolumn{2}{c}{\bf ŠIDÁK} & \multicolumn{2}{c}{\bf HOLM} & \multicolumn{2}{c}{\bf SIMES} 
    & \multicolumn{2}{c}{\bf HOCH} & \multicolumn{2}{c}{\bf BH} & \multicolumn{2}{c}{\bf -} \\ \hline
    {\bf 1} & 0&0476 & 0&0485 & 0&0472 & 0&0000 & 0&0472 & 0&0457 & 0&2051 \\ 
    {\bf 2} & 0&0009 & 0&0010 & 0&0014 & 0&0000 & 0&0013 & 0&0038 & 0&0213 \\ 
    {\bf 3} & 0&0000 & 0&0000 & 0&0000 & 0&0000 & 0&0000 & 0&0002 & 0&0012 \\ 
    {\bf 4} & 0&0000 & 0&0000 & 0&0000 & 0&0000 & 0&0000 & 0&0000 & 0&0000 \\ 
    {\bf 5} & 0&0000 & 0&0000 & 0&0000 & 0&0497 & 0&0000 & 0&0000 & 0&0000 \\ \hline
    {\bf $\sum$} & 0&0485 & 0&0495 & 0&0486 & 0&0497 & 0&0485 & 0&0497 & 0&2276 \\ 
  \end{tabular}
  \caption{Podiel počtu zamietnutých hypotéz v~každom opakovaní k počtu opakovaní, 
  pri~testovaní $5$~hypotéz rôznymi korekciami s~počtom opakovaní $100000$}
  \captionsetup{justification=centering}
  \label{tab04:2}
\end{table}

\begin{table}[h!]
  \centering
  \begin{tabular}{c|r@{.}lr@{.}lr@{.}lr@{.}lr@{.}lr@{.}lr@{.}l}
    & \multicolumn{2}{c}{\bf BONF} & \multicolumn{2}{c}{\bf ŠIDÁK} & \multicolumn{2}{c}{\bf HOLM} & \multicolumn{2}{c}{\bf SIMES} 
    & \multicolumn{2}{c}{\bf HOCH} & \multicolumn{2}{c}{\bf BH} & \multicolumn{2}{c}{\bf -} \\ \hline
    {\bf 1} & 0&0478 & 0&0486 & 0&0475 & 0&0000 & 0&0475 & 0&0458 & 0&3140 \\ 
    {\bf 2} & 0&0008 & 0&0009 & 0&0011 & 0&0000 & 0&0011 & 0&0306 & 0&0749 \\ 
    {\bf 3} & 0&0000 & 0&0000 & 0&0000 & 0&0000 & 0&0000 & 0&0031 & 0&0103 \\ 
    {\bf 4} & 0&0000 & 0&0000 & 0&0000 & 0&0000 & 0&0000 & 0&0000 & 0&0010 \\ 
    {\bf 5} & 0&0000 & 0&0000 & 0&0000 & 0&0000 & 0&0000 & 0&0000 & 0&0000 \\ 
    {\bf 6} & 0&0000 & 0&0000 & 0&0000 & 0&0000 & 0&0000 & 0&0000 & 0&0000 \\ 
    {\bf 7} & 0&0000 & 0&0000 & 0&0000 & 0&0000 & 0&0000 & 0&0000 & 0&0000 \\ 
    {\bf 8} & 0&0000 & 0&0000 & 0&0000 & 0&0000 & 0&0000 & 0&0000 & 0&0000 \\ 
    {\bf 9} & 0&0000 & 0&0000 & 0&0000 & 0&0000 & 0&0000 & 0&0000 & 0&0000 \\ 
    {\bf 10} & 0&0000 & 0&0000 & 0&0000 & 0&0497 & 0&0000 & 0&0000 & 0&0000 \\ \hline
    {\bf $\sum$} & 0&0486 & 0&0495 & 0&0486 & 0&0497 & 0&0486 & 0&0497 & 0&4002 \\ 
  \end{tabular}
  \caption{Podiel počtu zamietnutých hypotéz v~každom opakovaní k počtu opakovaní, 
  pri~testovaní $10$~hypotéz rôznymi korekciami s~počtom opakovaní $100000$}
  \captionsetup{justification=centering}
  \label{tab04:3}
\end{table}

V~Tabuľkách~\ref{tab04:1}, \ref{tab04:2}, \ref{tab04:3} vidíme podiel 
počtu  zamietnutých hypotéz v~každom opakovaní k počtu opakovaní 
pri~testovaní $3$, $5$, $10$~hypotéz.  
V~poslednom riadku je podiel počtu chýb, ktoré sme spravili pri~$100000$~opakovaniach 
k počtu všetkých opakovaní.  
Tým sme získali celkovú hladinu významnosti jednotlivých korekcií. 
Vidíme, že~každá korekcia dodržuje hladinu významnosti $\alpha=0.05$. 
Simesova korekcia kontroluje len~slabšiu verziu chyby ${\rm FWER}$, 
teda pri~platnosti všetkých hypotéz je~dodržaná hladina testu. 
Podľa hodnôt v~tabuľkách sa dá spozorovať, že~táto korekcia funguje iným spôsobom ako ostatné. 
Za~splnenia určitých podmienok zamieta všetky hypotézy, v~opačnom prípade nezamietne žiadnu. 
Benjamini-Hochbergova korekcia kontroluje chybu ${\rm FDR}$, ktorá nie~je až~taká striktná. 
V~porovnaní s~inými korekciami zamieta častejšie väčší počet hypotéz ako $1$-$2$. 
Ostatné korekcie kontrolujú silnejšiu verziu chyby ${\rm FWER}$, 
skutočná hladina týchto korekcií je~nižšia ako hladina Simesovej alebo Benjamini-Hochbergovej korekcie. 
Teda ich~sila by mala byť nižšia. 
Holmova a~Hochbergova korekcia majú podobné výsledky vzhľadom k~testovaniu každého počtu hypotéz, ktoré sme uviedli. 
Takisto si môžeme všimnúť, že počet chýb pri~použití Simesovej a~Benjamini-Hochbergovej korekcie je~rovnaký. 

\section{Z hľadiska sily testu}

V~tejto podkapitole budeme porovnávať silu korekcií za~rôznych podmienok. 
Sila testu závisí na~zvolenej alternatíve, na~počte nulových hypotéz 
a~v~našom prípade aj na~počte hypotéz, ktoré testujeme. 
Aby sme zistili silu testov, musíme vygenerovať dáta tak, aby neboli všetky nulové hypotézy platné. 
Z~tohto porovnania vylúčime Simesovu korekciu, 
ktorá kontroluje chybu ${\rm FWER}$ len v~slabšom zmysle, 
teda nie~je zaručená hladina významnosti $\alpha=0.05$. 

Rozoberieme viacero možností, budeme meniť pomer platných a~neplatných hypotéz, 
pričom neplatné hypotézy budeme generovať s~rôznymi strednými hodnotami. 
Silu testov porovnáme pre rôzne počty hypotéz, ktoré budeme testovať. 

Budeme predpokladať, že máme $K$ nezávislých náhodných výberov 
so~strednými hodnotami $\mu_i$ s~rozsahom $n=100$, kde $i \in \{ 1, \dots, K \}$. 
Dáta vygenerujeme z~normovaného normálneho rozdelenia. 
Budeme uvažovať niekoľko prípadov. 
Počet náhodných výberov $K$~bude postupne $4$, $8$, $16$, $32$, $64$, 
pričom budeme testovať $K$~hypotéz. 
Pomer neplatných hypotéz ku~všetkým hypotézam budeme meniť,  
rozdelíme ich na~4~rôzne prípady, 
počet nepatných hypotéz bude postupne ${K}$, $\frac{3}{4}K$, $\frac{1}{2}K$, $\frac{1}{4}K$. 
Ako sme spomenuli, sila testu často závisí na~rozdiele nulovej hypotézy a~testovaných dát. 
Preto budeme uvažovať $3$ možnosti vo~voľbe stredných hodnôt neplatných hypotéz. 

Budeme testovať strednú hodnotu náhodných výberov, 
testujeme $K$~hypotéz tvaru 
\begin{center} 
  $ H_1: \mu_1 = 0, $ \\
  $ \vdots $ \\
  $ H_K: \mu_K = 0. $ \\
\end{center}
Platné hypotézy vygenerujeme so~strednou hodnotou rovnou~$0$. 
Neplatné hypotézy budeme generovať postupne so~strednými hodnotami rovnými $0.1$, $0.2$, $0.3$. 

\begin{figure}[h!]
  \centering
  \includegraphics[width=\linewidth]{C:/Users/KIKA/Desktop/BP/R/obr3.pdf}
  \caption{Sila testovania pri~použití rôznych korekcií 
  so~strednými hodnotami neplatných hypotéz $0.1$, $0.2$, $0.3$ 
  a~podielom neplatných hypotéz ${1}$, $\frac{3}{4}$, $\frac{1}{2}$, $\frac{1}{4}$}
  \captionsetup{justification=centering}
  \label{obr04:1}
\end{figure}

Na~Obrázku~\ref{obr04:1} vidíme silu mnohonásobného testovanie s~použitím rozličných korekcií. 
Na~horizontálnej osy je počet hypotéz, ktoré sme testovali. 
Nad~obrázkami je stredná hodnota generovaných dát s~neplatnými hypotézami, 
napravo vidíme pomer neplatných hypotéz ku~všetkým hypotézam. 

Pre~neplatné hypotézy s~akoukoľvek strednou hodnotou platí
čím menší je počet neplatných hypotéz, tým väčšia je sila testu. 
Pri voľbe neplatnej hypotézy so~strednou hodnotou, ktorá je blízko strednej hodnoty z~nulovej hypotézy, 
je sila testu výrazne nižšia ako pri~iných stredných hodnotách. 
Je~to z~dôvodu, že test nevie vždy správne vyhodnotiť, či daná hypotéza platí, 
keďže sú dané stredné hodnoty blízko seba. 

Pri~voľbe strednej hodnoty $0.1$ pre~neplatné hypotézy nie~je medzi korekciami vidieť žiadny rozdiel 
pre~všetky zvolené pomery hypotéz. 
Pre~ostatné zvolené stredné hodnoty sú rozdiely vidieteľnejšie pri~vyššom počte neplatných hypotéz. 
Bonferroniho a~Šidákova majú veľmi podobnú silu testu vo~všetkých prípadoch. 
Holmova korekcia sa vo~väčšine prípadoch správa podobne ako Hochbergova, 
avšak v~prípade neplatnosti všetkých hypotéz má pri~nižsom počte hypotéz väčšiu silu. 
Najvyššiu silu vo~všetkých prípadoch má Benjamini-Hochbergova korekcia. 
Pri~zvyšujúcom počte neplatných hypotéz je sila tejto korekcie výrazne vyššia ako sily iných korekcií. 
Je~to kvôli kontrolovaniu chyby ${\rm FDR}$, ktorá nie~je až~taká striktná ako~${\rm FWER}$, 
ktorú kontrolujú všetky ostatné korekcie.

Na silu testu sa môžeme pozerať aj iným spôsobom. 
V predchádzajúcej podkapitole sme pomocou simulácii ukázali, že korekcie korigujú celkovú hladinu testovania. 
V tomto prípade by sme chceli overiť, že korekcie mnohonásobného testovanie nemajú vplyv na konzistenciu testu, 
teda chceme, aby sila testu konvergovala k $1$ pre rastúci rozsah náhodných výberov. 

Predpokladajme, že máme dva náhodné výbery 
$$ {\bf X}_1 = (X_{11}, X_{12}, \dots, X_{1n})^T, $$ 
$$ {\bf X}_2 = (X_{21}, X_{22}, \dots, X_{2n})^T, $$
z~rozdelení $N(\mu_1, \sigma^2_1)$ a~$N(\mu_2, \sigma^2_2)$. 
Aby sme ukázali konzistenciu, budeme meniť rozsah náhodných výberov $n$, 
postupne $50$, $100$, $150$, $200$, $250$. 
Budeme testovať tri hypotézy, ktoré budú mať tvar 
\begin{align*}
  H_1 & : \mu_1 = 0; \\
  H_2 & : \mu_2 = 0; \\ 
  H_3 & : \sigma^2_1 = \sigma^2_2. \\
\end{align*}
Na~zistenie sily testu, potrebujeme vygenerovať dáta, pre ktoré nebudú všetky hypotézy platné. 
Náhodné výbery, pre~ktoré budú platiť všetky hypotézy, budeme generovať z normovaného normálneho rozdelenia. 
V~tomto prípade uvedieme tri možnosti pre náhodné výbery, kde $a, b \in \R$ budú nami zvolené parametre:  
\begin{enumerate}
  \item ${\bf X}_1 \sim N(0,1)$ a ${\bf X}_2 \sim N(0,b)$, 
  v tejto možnosti budú hypotézy $H_1$, $H_2$ platné a  $H_3$ neplatná; 
  \item ${\bf X}_1 \sim N(0,1)$ a ${\bf X}_2 \sim N(a,b)$,
  hypotéza $H_1$ je platné, hypotézy $H_2$, $H_3$ sú neplatné; 
  \item ${\bf X}_1 \sim N(a,1)$ a ${\bf X}_2 \sim N(a,b)$,  
  v tomto prípade sú všetky hypotézy neplatné.  
\end{enumerate}
Poradie možností budeme používať v grafe.  
Keďže veľkosť sily závisí od zvolenej alternatívy, parametre $a$, $b$ budeme postupne meniť, 
uvedieme tri možnosti voľby týchto parametrov: 
\begin{itemize}
  \item $a=0.1$, $b=1.25$;
  \item $a=0.2$, $b=1.5$;
  \item $a=0.3$, $b=1.75$.
\end{itemize}

\begin{figure}[h!]
  \centering
  \includegraphics[width=\linewidth]{C:/Users/KIKA/Desktop/BP/R/obr4_portrait.pdf}
  \caption{Sila testovania pri~použití rôznych korekcií 
  s troma možnosťami voľby platných a neplatných hypotéz, 
  pre rôzne parametre rozdelení $a$, $b$}
  \captionsetup{justification=centering}
  \label{obr04:2}
\end{figure}

Na Obrázku \ref{obr04:2} je na horizontálnej osi uvedený rozsah náhodných výberov, 
na vertikálnej osi sila testu. 
Nad obrázkami vidíme zvolené parametre a napravo poradie možnosti ohľadne voľby generovaných náhodných výberov, 
ako sme uviedli vyššie.  
Vidíme, že pre každú voľbu parametrov $a$, $b$ 
a taktiež pre každú uvedenú možnosť voľby platných a neplatných hypotéz, 
sa sila testu zvyšuje s rastúcim rozsahom náhodného výberu. 
Pomocou Obrázka \ref{obr04:2} sme ukázali, že sila testu konverguje k $1$ 
pri rastúcom rozsahu náhodných výberov. 
Teda korekcie mnohonásobného testovania nemajú vplyv na konzistenciu testov. 




\chapter*{Záver}
\addcontentsline{toc}{chapter}{Záver}

Táto práca sa~zaoberala problémom s~hladinou testu pri~mnohonásobnom testovaní. 
Na~začiatku práce sme definovali kľúčové pojmy z~matematickej štatistiky, 
ako~p-hodnota, hladina a~sila testu. 

Podrobne sme objasnili problém s~hladinou testu nastávajúci pri~mnohonásobnom testovaní. 
Pomocou obrázkov sme ukázali rozdiely medzi testovaním jednej a~viacerých hypotéz. 
Definovali sme chyby ${\rm FWER}$ a~${\rm FDR}$, ktoré sa kontrolujú pri~testovaní 
a~opísali sme aký je medzi~nimi rozdiel. 

Táto práca sa zaoberala porovnávaním rozličných korekcií mnohonásobného testovania, 
vysvetlili sme ako fungujú vybrané korekcie. 
Pred~definovaním korekcií sme uviedli Booleovu nerovnosť, ktorá je základným tvrdením 
pre~Bonferroniho korekciu a~taktiež niektoré ďalšie odvíjajúce sa z~nej. 
Najskôr sme opísali simultánne zamietajúce korekcie, Bonferroniho a~Šidákovu. 
Porovnali sme upravenú hladinu testov pre~tieto dve~korekcie. 
Ostatné opísané korekcie patria medzi postupne zamietajúce. 
Definovali sme Holmovu a~Simesovu korekciu, z~ktorých bola odvodená Hochbergova korekcia. 
Ako poslednú korekciu sme uviedli Benjamini-Hochbergovu odlíšujúcu sa od~ostatných 
kontrolou chyby ${\rm FDR}$ narozdiel od~kontroly ${\rm FWER}$. 
Pre každú korekciu sme uviedli základné predpoklady na~jej definíciu 
a~opísali sme postup zamietania hypotéz. 
Pri väčšine z~nich sme uviedli dôkaz, že korekcie naozaj kontrolujú dané chyby. 

V~praktickej časti práce sme porovnali korekcie najskôr z~hľadiska hladiny testu. 
Pomocou simulácii sme určili počet zamietnutých platných hypotéz na~určitý počet opakovaní 
pre~testovanie rôzneho počtu hypotéz. 
Následné sme pomocou týchto hodnôt zistili celkovú hladinu testu pre každú korekciu.
Podľa~výsledkov simulácii sme porovnali korekcie a~vysvetlili v~čom sa odlišujú. 
Ďalšie hľadisko porovnávania bola sila testu. 
Keďže sila testu závisí od~rôznych faktorov, v~simuláciach sme menili podmienky pre~testovanie, 
konkrétne pomer neplatných a~platných hypotéz a~strednú hodnotu vygenerovaných dát neplatných hypotéz. 
Simulácie sme previedli pre~rôzne počty testujúcich hypotéz. 
V~niektorých prípadoch sme nepostrehli rozdiely v~korekciách, 
avšak pri~väčšom počte neplatných hypotéz bola sila testu Benjamini-Hochbergovej korekcie výrazne vyššia 
ako sila testu pri~použítí iných korekcií. 
Je~to~práve z~dôvodu kontroly chyby ${\rm FDR}$, ktorá nie je až~striktná ako~kontrola ${\rm FWER}$. 
Na konci praktickej časti sme pomocou simulácii ukázali silu testu pre testovanie troch hypotéz 
s rôznymi rozsahmi náhodného výberu. 
Uvažovali sme rôzne možnosti platných a neplatných hypotéz a takisto sme menili parametre rozdelení náhodných výberov, 
pre ktoré hypotézy nie sú platné. 
Touto simuláciou sme ukázali, že sila testu sa v každom prípade zvyšuje pre rastúci rozsah náhodných výberov. 
To znamená, že korekcie nemajú vplyv na konzistenciu testu. 
 

%%% Seznam použité literatury
\include{literatura}

%%% Obrázky v bakalářské práci
%%% (pokud jich je malé množství, obvykle není třeba seznam uvádět)
%!\listoffigures

%%% Tabulky v bakalářské práci (opět nemusí být nutné uvádět)
%%% U matematických prací může být lepší přemístit seznam tabulek na začátek práce.
%!\listoftables

%%% Použité zkratky v bakalářské práci (opět nemusí být nutné uvádět)
%%% U matematických prací může být lepší přemístit seznam zkratek na začátek práce.
%!\chapwithtoc{Zoznam použitých skratiek}

%%% Přílohy k bakalářské práci, existují-li. Každá příloha musí být alespoň jednou
%%% odkazována z vlastního textu práce. Přílohy se číslují.
%%%
%%% Do tištěné verze se spíše hodí přílohy, které lze číst a prohlížet (dodatečné
%%% tabulky a grafy, různé textové doplňky, ukázky výstupů z počítačových programů,
%%% apod.). Do elektronické verze se hodí přílohy, které budou spíše používány
%%% v elektronické podobě než čteny (zdrojové kódy programů, datové soubory,
%%% interaktivní grafy apod.). Elektronické přílohy se nahrávají do SISu a lze
%%% je také do práce vložit na CD/DVD. Povolené formáty souborů specifikuje
%%% opatření rektora č. 72/2017.
%!\appendix
%!\chapter{Prílohy}

%!\section{Prvá príloha}

\openright
\end{document}
