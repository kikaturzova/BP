\chapter*{Úvod}
\addcontentsline{toc}{chapter}{Úvod}

Testovanie hypotéz umožňuje posúdiť, či získané dáta z experimentu vyhovujú vopred určenému predpokladu. 
Pri testovaní hypotéz je dôležité správne určiť nulovú a alternatívnu hypotézu. 
Tieto dve hypotézy sú disjunktné tvrdenia, pričom chceme overiť, ktoré z nich platí. 
Testujeme vždy nulovú hypotézu, ktorú buď zamietneme v prospech alternatívy, alebo nezamietneme. 
V druhom prípade nevieme určiť, ktorá hypotéza je platná. 
Test však nemusí vždy rozhodnúť správne. 
Existujú dva typy chýb vyskytujúcich sa pri testovaní $-$ zamietnutie platnej hypotézy, chyba 1. druhu, a 
nezamietnutie neplatnej hypotézy, chyba 2. druhu. 
Za závažnejšiu je považovaná chyba 1. druhu, preto chceme, aby pravdepodobnosť zamietnutia platnej hypotézy bola čo najmenšia. 
Táto pravdepobnosť sa nazýva hladina testu a je potrebné ju určiť na začiatku každého testovania. 
Okrem hladiny testu nás zaujíma taktiež sila testu, 
čo je pravdepodobnosť zamietnutia neplatnej hypotézy. 
Platí vzťah, že čím menšia je pravdepobnosť nezamietnutia neplatnej hypotézy, tým väčšia je sila testu. 

Mnohonásobné testovanie je simultánne testovanie väčšieho počtu hypotéz. 
Dôvodom testovania viacerých hypotéz môže byť závislosť hypotéz, ktoré chceme testovať. 
V tomto prípade sa opäť snažíme zamietnuť čo najmenej platných hypotéz, avšak so zvyšujúcim sa počtom hypotéz sa zvyšuje pravdepodobnosť chyby 1.~druhu. 
Aby bola dodržaná hladina testu, je potrebné upraviť hladinu jednotlivých testov, 
prípadne upraviť kritériá na zamietanie hypotéz. 

Cieľom práce je porovnať vybrané korekcie mnohonásobného testovania, konkrétne Bonferroniho, 
Šidákovu, Holmovu, Simesovu, Hochbergovu a~Benjamini-Hochbergovu korekciu. 
V práci budeme predpokladať čitateľovu znalosť základných pojmov z pravdepodobnosti a matematickej štatistiky. 
V prvej kapitole vysvetlíme niektoré pojmy z matematickej štatistiky, ktoré budú kľúčové v celej práci. 
Vysvetlíme, čo je mnohonásobné testovanie a podrobnejšie ukážeme problém s hladinou významnosti, 
ktorý pri ňom nastáva. 
Taktiež definujeme niektoré chyby, ktoré nastanú zamietnutím platných hypotéz. 
Je viacero spôsobov, ako sa dá na tieto chyby pozerať. 
My vyberáme tie z nich, ktoré sú pri mnohonásobnom testovaní kontrolované najčastejšie a ukážeme vzťah medzi nimi.
Druhá kapitola sa zaoberá spomínanými korekciami, 
pričom pre každú z nich je vysvetlený postup zamietania hypotéz 
a overenie, či kontrolujú definované chyby. 
V tretej kapitole budeme porovnávať korekcie pomocou simulácií z hľadiska hladiny a sily testu. 

