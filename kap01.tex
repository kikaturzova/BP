\chapter{Mnohonásobné testovanie}

V tejto práci budeme predpokladať, že čitateľ pozná základné pojmy z matematickej štatistiky ako pravdepodobnostný a parametrický priestor, náhodný výber, 
model, testová štatistika a kritický obor.
V prvej kapitole definujeme pojmy z matematickej štatistiky, ktoré pre nás budú v tejto práci klúčové, 
vysvetlíme ako funguje mnohonásobné testovanie a objasníme aké problémy nastávajú pri jeho používaní. 
Definujeme chyby, ktoré budeme kontrolovať pri mnohonásobnom testovaní. 

\section{Základné pojmy}

V tejto podkapitole budeme predpokladať, že máme náhodný výber $${\bf X} = (X_1, X_2, \dots, X_n)^T$$ 
z~rozdelenia $F \in \mathcal{F}$, kde $\mathcal{F} = \{F(\theta), ~ \theta \in \Theta\}$ je model. 
Teda $\Theta$ je množina všetkých možných hodnôt parametru $\theta$ v modele $\mathcal{F}$. 
Skutočný parameter budeme značiť~$\theta_X$. 
Označme $\Theta_H$ a $\Theta_A$ dve disjunktné podmnožiny parametrického priestoru $\Theta$. 

\begin{definicia}\label{def1}
  Množinu $\Theta_H$ nazývame nulová hypotéza a $\Theta_A$ alternatívna hypotéza. 
  Množinu všetkých rozdelení z modelu $\mathcal{F}$, ktorých parametre splňajú nulovú hypotézu, budeme značiť $\mathcal{F}_H$, 
  podobne množinu rozdelení s parametrami spĺňajúcimi alternatívnu hypotézu značíme $\mathcal{F}_A$. 
\end{definicia}  

Pri testovaní skutočného parametru $\theta_X$ budeme zapisovať nulovú hypotézu $H: \theta_X \in \Theta_A$ 
proti alternatíve $A: \theta_X \in \Theta_A$. 

Testovanie hypotéz sa vyhodnocuje pomocou štatistického testu, 
na jeho definovanie potrebujeme nasledujúce pojmy. 
Testová štatistika $\mathcal{S} : \R^n \mapsto \Theta$ je merateľná funkcia dát náhodného výberu ${\bf X}$, 
ktorú volíme tak, aby sme vedeli určiť jej presné alebo asymptotické rozdelenie. 
Následne podľa tohto rozdelenia určíme kritický obor $\mathcal{C}$, 
pomocou ktorého vyhodnotíme test. 

Nech $\mathcal{S}$ je testová štatistika a $\mathcal{C}$ je kritický obor.  
Ak $\mathcal{S} \in \mathcal{C}$, zamietame nulovú hypotézu v prospech alternatívnej hypotézy.  
Ak $\mathcal{S} \notin \mathcal{C}$, nemôžeme zamietnuť nulovú hypotézu v prospech alternatívnej hypotézy. 

Štatistický test nemusí v každom prípade rozhodnúť správne. 
Preto nastávajú štyri rôzne situácie vzhľadom k platnosti nulovej hypotézy a vyhodnotenia testu, 
ktoré môžeme prehľadne vidieť v~Tabuľke \ref{tab01:1}. 

\begin{table}[h!]
  \centering
  \begin{tabular}{|c|c|c|}
    \hline
     & Platí $H$ & Neplatí $H$ \\ \hline
    Zamietame $H$ & Chyba 1.druhu & OK \\ \hline
    Nezamietame $H$ & OK & Chyba 2.druhu \\ \hline
  \end{tabular}
  \caption{Všetky možnosti, ktoré nastávajú pri testovaní hypotéz}
  \captionsetup{justification=centering}
  \label{tab01:1}
\end{table}

V prípade zamietnutia platnej hypotézy hovoríme o chybe 1. druhu. 
Nezamietnutie neplatnej hypotézy sa nazýva chyba 2. druhu. 
Test volíme tak, aby chyba 1. druhu bola závažnejšia. 
Preto budeme kontrolovať jej pravdepodobnosť a budeme chcieť, aby bola čo najmenšia.

Na kontrolu chyby 1. druhu sa používa hladina testu. Niekedy budeme používať takisto výraz hladina významnosti. 

\begin{definicia}\label{def3}
  Nech $\alpha \in (0,1)$ je vopred dané číslo, nech máme test s testovou štatistikou $\mathcal{S}$ a kritickým oborom $\mathcal{C}$. 
  Hladina testu je rovná $\alpha$, ak je splnená podmienka $$ \sup P_H (\mathcal{S} \in \mathcal{C}) = \alpha, $$
  kde $P_H$ označuje pravdepodobnosť za predpokladu platnosti nulovej hypotézy $H$ 
  a supremum uvažujeme vzhľadom k~všetkým nulovým hypotézam. 
  V~niektorých prípadoch je hladina testu dosiahnutá len asymptoticky, potom musí pre~$n~\longrightarrow~\infty$ platiť upravená podmienka 
  $$ \sup \lim_{n \rightarrow \infty} P_H (\mathcal{S} \in \mathcal{C}) = \alpha. $$
\end{definicia}  

Hladina testu je pravdepodobnosť zamietnutia nulovej hypotézy, ktorá v skutočnosti platí. 
Je to teda pravdepodobnosť chyby 1. druhu. 
Pri testovaní vždy na začiatku určíme hladinu významnosti daného testu. 
V tejto práci budeme väčšinou voliť hladinu testu $\alpha=0.05$. 

Nech $\beta (A) = P_A (\mathcal{S} \in \mathcal{C})$, 
kde $P_A$ označuje pravdepodobnosť za predpokladu platnosti alternatívnej hypotézy. 
Hodnotu $\beta (A)$ nazývame sila testu a je to pravdepodobnosť zamietnutia neplatnej hypotézy. 
Pri testovaní chceme, aby sila testu bola čo najvyššia. 
V štvrtej kapitole budeme porovnávať korekcie mnohonásobného testovania vzhľadom k ich sile. 

\section{Definícia mnohonásobného testovania}

V tejto práci budeme predpokladať, že máme $K$ hypotéz, ktoré chceme testovať, 
nulové hypotézy budeme značiť $H_i$ a k nim príslušné
testové štatistiky $\mathcal{S}^{(i)}$ a~kritické obory $\mathcal{C}^{(i)}$, $i \in \{1, \dots, K\}$. 
Množinu všetkých nulových hypotéz budeme značiť ${\mathcal{H}} = \{ H_1, \dots, H_K \}$. 
Všetky nulové hypotézy budeme testovať simultánne a~nezávisle na sebe, 
tento typ testovania sa nazýva mnohonásobné testovanie. 
V~celej práci budeme predpokladať, že testované hypotézy sú navzájom nezávislé. 

Ako sme spomínali v predchádzajúcej podkapitole, pri testovaní hypotéz je potrebné kontrolovať hladinu testu, ktorú určíme na začiatku. 
Pre mnohonásobné testovanie zvolíme hladinu testu $\alpha$ a každú hypotézu $H_i$ budeme testovať na tejto hladine. 
To znamená, že pravdepodobnosť zamietnutia platnej nulovej hypotézy $H_i$ bude rovná $\alpha$ pre každé $i \in \{1, \dots, K\}$
$$ P_{H_i} \left( \mathcal{S}^{(i)} \in \mathcal{C}^{(i)} \right) = \alpha. $$
Takisto vieme vyjadriť pravdepodobnosť nezamietnutia platnej hypotézy, teda pravdepodobnosť, že nespravíme chybu 1. druhu 
$$ P_{H_i} \left( \mathcal{S}^{(i)} \notin \mathcal{C}^{(i)} \right) = 1 - \alpha. $$ 

\section{Problémy s hladinou testu}

??? Pri mnohonásobnom testovaní nestačí kontrolovať hladinu jednotlivých testov. 
Namiesto toho budeme kontrolovať pravdepodobnosť zamietnutia aspoň jednej platnej hypotézy, 
teda pravdepodobnosť zjednotenia všetkých zamietnutí hypotéz 
za predpokladu platnosti všetkých nulových hypotéz. 
Túto pravdepodobnosť budeme označovať~$\alpha_K$ a formálne to môžeme zapísať nasledovným spôsobom 
$$ P_{\mathcal{H}_0} \left( \bigcup_{i=1}^{K} \left[ \mathcal{S}^{(i)} \in \mathcal{C}^{(i)} \right] \right) = \alpha_K, $$
kde $\mathcal{H}_0 = {\bigcap_{i=1}^{K} H_i}$ označuje skutočnosť, že všetky nulové hypotézy sú platné. 
Podobne ako sme opísali v predchádzajúcej podkapitole, môžeme vyjadriť pravdepodobnosť, 
že nespravíme chybu 1. druhu ani pri jednom testovaní hypotézy $H_i$, kde $i \in \{1, \dots, K\}$ 
$$ P_{\mathcal{H}_0} \left( \bigcap_{i=1}^{K} \left[ \mathcal{S}^{(i)} \notin \mathcal{C}^{(i)} \right] \right) 
= \prod_{i=1}^{K} P_{H_i} \left( \mathcal{S}^{(i)} \notin \mathcal{C}^{(i)} \right)
= (1 - \alpha)^K. $$ 
Pravdepodobnosť zamietnutia aspoň jednej platnej hypotézy sa dá zapísať nasledujúcim spôsobom
$$ P_{\mathcal{H}_0} \left( \exists i \in \{1, \dots, K\}: \left[ \mathcal{S}^{(i)} \in \mathcal{C}^{(i)} \right] \right) 
   = 1 - P_{\mathcal{H}_0} \left( \bigcap_{i=1}^{K} [\mathcal{S}^{(i)} \notin \mathcal{C}^{(i)}] \right) = 1 - (1 - \alpha)^K. $$ 
Potom platí 
\begin{align*} 
\alpha_K = P_{\mathcal{H}_0} \left( \bigcup_{i=1}^{K} \left[ \mathcal{S}^{(i)} \in \mathcal{C}^{(i)} \right] \right)
& =  P_{\mathcal{H}_0} \left( \exists i \in \{1, \dots, K\}: \left[ \mathcal{S}^{(i)} \in \mathcal{C}^{(i)} \right] \right) \\
& = 1 - (1 - \alpha)^K > \alpha. \\
\end{align*}
Teda $\alpha_K$ je väčšia ako $\alpha$ pre každé $K > 1$ a $\alpha \in (0,1)$.
Z toho vyplýva, že pri testovaní viacerých hypotéz je pravdepodobnosť zamietnutia aspoň jednej platnej hypotézy 
väčšia ako hladina významnosti $\alpha$, ktorú sme určili na začiatku. 
Aby bola splnená celková hladina testu $\alpha$, je potrebné upraviť hladinu jednotlivých testov. 
Ukážeme to takisto pomocou obrázkov. 
Na testovanie hypotéz budeme používať p-hodnotu, ktorú teraz definujeme. 

\begin{definicia}\label{def4}
  Nech $\mathcal{S}$ je testová štatistika, $\mathcal{C}$ je kritický obor a nech $s$ je hodnota testovej štatistiky $\mathcal{S}$ 
  spočítaná z napozorovaných dát, ktoré chceme testovať. 
  \mbox{P-hodnotu} definujeme ako 
  \begin{itemize} 
    \item $ p = \sup P_H (\mathcal{S} \leq s) $, 
    ak $\mathcal{C} = \langle c_U, \infty)$ pre nejaké $c_U \in \R$;
    \item $ p = \sup P_H (\mathcal{S} \geq s) $, 
    ak $\mathcal{C} = (-\infty, c_L \rangle$ pre nejaké $c_L \in \R$;
    \item $ p = \sup 2 \min \{P_H (\mathcal{S}) \leq s), P_F (\mathcal{S}) \geq s)\} $, 
    ak $\mathcal{C} = (-\infty, c_L \rangle \cup \langle c_U, \infty)$  
    pre~nejaké $c_L, c_U \in \R$, $c_L < c_U$ a zároveň je splnená podmienka 
    \newline $ \sup P_H (\mathcal{S}) \leq c_L) 
    = \sup P_H (\mathcal{S}) \geq c_U) = \frac{\alpha}{2}. $
  \end{itemize}
  Supremum uvažujeme vzhľadom k všetkým nulovým hypotézam. 
\end{definicia}  

P-hodnota sa niekedy nazýva dosiahnutá hladina testu. 
Podmienka na konci definície musí byť splnená, aby celková hladina testu bola rovná $\alpha$. 
P-hodnota je takisto kľučový pojem na definovanie korekcií, ktoré uvedieme v nasledujúcej kapitole. 

Ako sme zmienili vyššie, pomocou p-hodnoty vieme rozhodnúť, či máme hypotézu zamietnuť alebo nie. 
Budeme k tomu potrebovať nasledujúce tvrdenie, ktorého dôkaz viď \cite[Tvrdenie 4.1]{Omelka19}. 

\begin{tvrd}\label{tvrd01}
  Nech $\mathcal{S}$ je testová štatistika so spojitým rozdelením 
  a~$p$~je p-hodnota. 
  Uvažujme test hypotézy $H$ proti alternatíve $A$ daný pravidlom 
  \begin{center}
    $H$ zamietame $\Longleftrightarrow$ $p \leq \alpha$, \\
    $H$ nezamietame $\Longleftrightarrow$ $p > \alpha$.\\
  \end{center}  
  Potom má tento test hladinu $\alpha$. 
\end{tvrd}  

Na začiatku ukážeme aká je skutočná hladina významnosti v~prípade testovania jednej hypotézy. 
Všetky hypotézy budeme testovať na~hladine $\alpha = 0.05$. 
V~celej podkapitole budeme predpokladať, že máme dva nezávislé náhodné výbery
$$ {\bf X}_1 = (X_{11}, X_{12}, \dots, X_{1n})^T, $$ 
$$ {\bf X}_2 = (X_{21}, X_{22}, \dots, X_{2n})^T, $$
z~rozdelení $N(\mu_1, \sigma^2)$ a~$N(\mu_2, \sigma^2)$, 
teda budeme pracovať s modelom 
$$ \mathcal{F} = \{ N(\mu, \sigma^2),~\mu \in \R,~\sigma^2>0 \}. $$ 
Rozsah náhodných výberov bude $n=100$. 
Oba náhodné výbery vygenerujeme z~normovaného normálneho rozdelenia. 

Najskôr budeme testovať hypotézu $H_1$, ktorá bude tvaru 
$$ H_1: \mu_1 = 0. $$
Presným jednovýberovým t-testom budeme testovať vygenerované dáta. 
Tento postup budeme opakovať niekoľko krát po sebe, pričom počet testovaní budeme meniť, 
postupne $100, 1000, 10000, 100000, 1000000$. 
Potrebujeme zistiť počet zamietnutých platných hypotéz. 
Všetky hypotézy sú platné a podiel zamietnutých hypotéz k počtu testovaní bude rovný skutočnej hladine testu. 

V druhom prípade budeme zároveň testovať dve hypotézy, jedna bude testovať strednú hodnotu 
a druhá rozptyl náhodného výberu ${\bf X}_1$. 
Hypotézy budú tvaru
\begin{align*}
H_1 & : \mu_1 = 0; \\
H_2 & : \sigma^2 = 1. \\
\end{align*}

V treťom prípade budeme testovať naviac jednu hypotézu, 
rovnosť stredných hodnôt náhodných výberov ${\bf X}_1$, ${\bf X}_2$. 
Hypotézy budú mať tvar 
\begin{align*}
H_1 & : \mu_1 = 0; \\
H_2 & : \sigma^2 = 1; \\ 
H_3 & : \mu_1 = \mu_2. \\
\end{align*}

Mnohonásobné testovanie zopakujeme niekoľko krát po sebe pre rôzne nagenerované dáta, 
počet opakovaní bude rovnaký ako predtým. 
Dáta boli vygenerované tak, aby boli hypotézy platné. 
Strednú hodnotu budeme testovať presným jednovýberovým t-testom a rozptyl pomocou jednovýberového $\chi^2$-testu. 
Rovnosť stredných hodnôt otestujeme presným dvojvýberovým t-testom, 
je splnený predpoklad rovnosti rozptylov. 

\begin{figure}[h!]
  \centering
  \includegraphics[width=\linewidth]{C:/Users/KIKA/Desktop/BP/R/obr2.pdf}
  \caption{Skutočná hladina významnosti pri testovaní rôzneho počtu hypotéz 
  pri počtoch opakovaní $100, 1000, 10000, 100000, 1000000$}
  \captionsetup{justification=centering}
  \label{obr02:2}
\end{figure}

Na Obrázku \ref{obr02:2} je porovnanie skutočnej hladiny významnosti pri testovaní jednej hypotézy 
a pri mnohonásobnom testovaní dvoch alebo troch hypotéz. 
Už~pri testovaní malého počtu hypotéz je skutočná hladina významnosti výrazne vyššia 
ako hladina, ktorú sme určili na začiatku. 

\section{Definície chýb FWER a FDR}

Ako sme ukázali v predchádzajúcej podkapitole, pri mnohonásobnom testovaní 
je potrebné kontrolovať chyby, ktoré sú prísnejšie ako je chyba 1. druhu. 
V~tejto podkapitole definujeme Familywise Error Rate a False Discovery Rate, 
budeme ich kontrolovať v korekciách, ktoré uvedieme v tejto práci. 

Nech $\mathcal{H}'$ je podmnožina $\mathcal{H}$. 
Skutočnosť, že platia všetky hypotézy v $\mathcal{H}'$ označíme ${\mathcal{H}}'_0$. 

\begin{definicia}\label{def5} 
  Nech máme $K$ hypotéz $H_1, \dots, H_K$ a nech $\mathcal{S}^{(i)}$ sú ich testové štatistiky 
  a $\mathcal{C}^{(i)}$ kritické obory, $i \in \{1, \dots, K\}$. 
  Familywise Error Rate definujeme ako pravdepodobnosť zamietnutia aspoň jednej platnej hypotézy za predpokladu, 
  že platia hypotézy z~ktorejkoľvek podmnožiny $\mathcal{H}' \subseteq \mathcal{H} = \{ H_1, \dots, H_K \}$.  
  Budeme ju značiť ${\rm FWER}$. 
  Teda platí 
  $$ {\rm FWER} = P_{\mathcal{H}'_0} \left( \exists H_i \in {\mathcal{H}}'_0: \left[ \mathcal{S}^{(i)} \in \mathcal{C}^{(i)} \right] \right). $$
\end{definicia}

Chyba ${\rm FWER}$ sa dá kontrolovať dvomi rôznymi spôsobmi, 
v Definícii \ref{def5} sme uviedli silnú kontrolu chyby ${\rm FWER}$. 
Pri druhom spôsobe predpokladáme platnosť všetkých nulových hypotéz, ktoré testujeme, 
ako sme uviedli v predcházajúcej podkapitole 
$$ {\rm FWER} = P_{\mathcal{H}_0} \left( \exists i \in \{1, \dots, K\}: \left[ \mathcal{S}^{(i)} \in \mathcal{C}^{(i)} \right] \right). $$
Nazýva sa to slabá kontrola chyby ${\rm FWER}$. 
V tomto prípade máme zaručenú ${\rm FWER}$ na hladine $\alpha$ len za platnosti všetkých hypotéz. 
Preto je výhodnejšie kontrolovať silnú kontrolu chyby ${\rm FWER}$. 

Pri mnohonásobnom testovaní budeme väčšinou kontrolovať Familywise Error Rate a budeme chcieť, 
aby bola táto chyba menšia ako zvolená hladina významnosti. 
V tejto práci budeme pracovať so silnou kontrolou chyby ${\rm FWER}$, 
pokiaľ explicitne neuvedieme inak.  

Problém s mnohonásobným testovaním ukážeme na príklade s konkrétnymi hodnotami. 
Nech $\alpha = 0.05$ a počet hypotéz $K = 10$. Jednotlivé hypotézy budeme testovať na hladine $\alpha$. 
Pravdepodobnosť, že spravíme minimálne jednu chybu pri~mnohonásobnom testovaní, 
bude ${\rm FWER} = 1 - (1 - 0.05)^{10} = 0.4012631$. 

\begin{figure}[h!]
  \centering
  \includegraphics[width=\linewidth]{C:/Users/KIKA/Desktop/BP/R/obr1.pdf}
  \captionsetup{justification=centering}
  \caption{Familywise Rrror Rate pri rôznych počtoch hypotéz s rôznymi hodnotami $\alpha$}
  \label{obr02:1}
\end{figure}

Na Obrázku \ref{obr02:1} vidíme, že už pri nižšom počte hypotéz je veľkosť ${\rm FWER}$ 
výrazne vyššia ako hladina významnosti, ktorú sme chceli dodržať. 

Aby bola táto chyba menšia alebo rovná ako $\alpha$, ktoré sme zvolili na začiatku, 
je nutné testovať jednotlivé hypotézy s menšou hladinou významnosti. 

V Tabuľke \ref{tab02:1} môžeme vidieť všetky možnosti, ktoré môžu nastať pri mnohonásobnom testovaní, 
pričom $D$, $E$, $F$, $G$ označujú počty hypotéz $H_i$ v každej možnosti pre $i \in \{1, \dots, K\}$, kde $K$ je ich súčet. 
$K_0$ označuje počet platných hypotéz a $Z$ počet zamietnutých hypotéz. 
Nás bude najviac zaujímať veľkosť $D$, pretože ide o~počet zamietnutých platných hypotéz. 

\begin{table}[h!]
    \centering
    \begin{tabular}{|c|c|c|c|}
      \hline
       & Platné $H_i$ & Neplatné $H_i$ & Celkom \\ \hline
      Zamietnuté $H_i$ & $D$ & $E$ & $Z$ \\ \hline
      Nezamietnuté $H_i$ & $F$ & $G$ & $K$-$Z$ \\ \hline
      Celkom & $K_0$ & $K$-$K_0$ & $K$ \\ \hline
    \end{tabular}
    \captionsetup{justification=centering}
    \caption{Počet hypotéz v každej možnosti, ktorá nastáva}
    \label{tab02:1}
\end{table}

Familywise Error Rate sa dá zapísať aj pomocou tohto značenia, platí 
$$ {\rm FWER} = P (D \geq 1). $$ 
Pri niektorých korekciách sa často kontroluje chyba ${\rm FDR}$, ktorá nie je taká striktná ako ${\rm FWER}$. 

\begin{definicia}\label{def6}
  False Discovery Rate definujeme ako strednú hodnotu z podielu zamietnutých platných hypotéz k zamietnutým hypotézam  
  za predpokladu zamietnutia aspoň jednej hypotézy. Budeme ju značiť ${\rm FDR}$. 
  Platí 
  $$ {\rm FDR} = E \left( \frac{D}{Z} ~ \bigg| ~ Z>0 \right) P(Z>0). $$
\end{definicia}

Ak sú všetky nulové hypotézy pravdivé, ${\rm FWER}$ a ${\rm FDR}$ sa rovnajú, dôkaz nájdeme v článku \cite{Benjamini&Hochberg95}. 
V ostatných prípadoch platí ${\rm FDR} \leq {\rm FWER}$. 
Pokiaľ kontrolujeme Familywise Error Rate, kontrolujeme zároveň aj False Discovery Rate. 
Opačná implikácia neplatí. 
