Mnohonásobné testovanie sa používa výhradne na kvantitatívne dáta. 
Na začiatku predpokladajme, že máme $k$ nezávislých náhodných výberov, kde $k \geq 2$. 
Pomocou mnohonásobného testovania chceme zistiť, či sa rovnajú stredné hodnoty všetkých náhodných výberov 
a taktiež ktoré náhodné výbery majú odlišnú strednú hodnotu. 

\begin{center} 
$ {\bf X}_1 = (X_{11}, X_{12}, \dots, X_{1n_1}) $ s rozdelením $F_1$ \\
$ {\bf X}_2 = (X_{21}, X_{22}, \dots, X_{1n_2}) $ s rozdelením $F_2$ \\
$\vdots$ \\
$ {\bf X}_k = (X_{k1}, X_{k2}, \dots, X_{kn_k}) $ s rozdelením $F_k$ \\
\end{center}

Náhodný výber ${\bf X}_i$ je vytvorený z $n_i$ nezávislých rovnako rozdelených náhodných veličín, 
ktoré pochádzajú z rozdelenia $F_i$, kde $i=1, \dots, k$.
Označme $N = \sum_{i=1}^{k} n_i$ počet všetkých náhodných veličín. 

Ak by bolo $k=2$, vieme otestovať hypotézy pomocou dvojvýberových testov. 
V tomto prípade teda nie je potrebné mnohonásobné porovnávanie. 

Ak by sme len chceli zistiť, či sú stredné hodnoty všetkých náhodných výberov rovnaké, 
je možné použiť jednoduché triedenie, ktoré sa nazýva taktiež analýza roptylov. 
V tomto prípade však musíme predpokladať, že náhodné výbery pochádzajú z normálneho rozdelenia 
so zhodnými rozptylmi. 
Budeme teda uvažovať model
$$ \mathcal{F} = \{ F_i = N(\mu_i, \sigma^2),~\mu_i \in \R,~\sigma^2 \in \R,~\sigma^2>0,~i \in \{1, \dots, k\} \}. $$

Keďže nulová hypotéza je rovnosť všetkých stredných hodnôt, 
na jej zamietnutie stačí nájsť dva rôzne náhodné výbery s rozdielnou strednou hodnotou. 
Hypotézy budú vyzerať nasledovne
$$ H_0: \mu_1 = \mu_2 = \dots = \mu_k, $$
$$ H_1: \exists~i, j \in \{1, \dots, k\},~i \neq j : \mu_i \neq \mu_j. $$ 

Použijeme testovú štatistiku, ktorá bude mať tvar 
$$ F_A = \frac{\frac{SS_A}{k-1}}{\frac{SS_e}{N-k}}, $$
$$ SS_A = \sum_{i=1}^{k} n_i ( \overline{Y}_{i+} - \overline{Y}_{++} )^2,~
SS_e = \sum_{i=1}^{k} \sum_{j=1}^{n_i} ( Y_{ij} - \overline{Y}_{i+} )^2. $$
$Y_{ij}$ sú náhodné veličiny z náhodného výberu, 
$\overline{Y}_{i+}$ sú priemery jednotlivých náhodných výberov
a $\overline{Y}_{++}$ je celkový priemer všetkých náhodných veličín z náhodných výberov. 
$SS_A$ sa nazýva súčet štvorcov skupín a $SS_e$ reziduálny súčet štvorcov. 

Odvodenie testovej štatistiky je pomerne zložité a nie je to cieľom tejto práce, 
preto ho nebudeme opisovať podrobnejšie. Je možné ho vidieť v \cite{Andel07}. 

Spôsobom, ktorý sme opísali vyššie je môžné zistiť rovnosť stredných hodnôt všetkých náhodných výberov, 
avšak žiadnu ďalšiu informáciu nezískame. 
Ak by sme chceli zistiť konkrétnejšie informácie o dátach, napríklad zaradiť do skupín náhodné výbery s rovnakými strednými hodnotami, 
prípadne zistiť ktorý náhodný výber má odlišnú strednú hodnotu od ostatných, musíme postupovať inak. 
Jednou z možností je otestovať stredné hodnoty každých dvoch náhodných výberov. 
Z $k$ náhodných výberov vytvoríme $\frac{k(k-1)}{2}$ dvojíc, toto číslo označíme $M$. Teda budeme mať $M$ hypotéz, ktoré budeme testovať. 

Nulové a alternatívne hypotézy budeme označovať postupne $H^{(i)}_0$ a $H^{(i)}_1$, kde $i \in \{1, \dots, M\}$. 
Testové štatistiku budeme značiť $S^{(i)}$, kritický obor $C^{(i)}$ a štatistický test zložený z testovej štatistiky  $S^{(i)}$ 
a kritického oboru $C^{(i)}$ budeme značiť $T^{(i)}$. 
Testovanie hypotéz prebehne tak ako sme opisovali v prvej kapitole, 
hypotézu $H^{(i)}_0$ budeme zamietať v prospech $H^{(i)}_1$, ak $S^{(i)} \in C^{(i)}$ a naopak, kde $i \in \{1, \dots, M\}$. 






V tejto podkapitole pomocou obrázkov ukážeme, že $\alpha_K$ je naozaj väčšia ako $\alpha$. 
V našom prípade zvolíme $\alpha = 0,05$. 

Nech ${\bf X} = (X_1, X_2, \dots, X_n)^T$ je náhodný výber s rozdelením $N(\mu, \sigma^2)$, 
teda budeme pracovať s modelom $\mathcal{F} = \{ N(\mu, \sigma^2),~\mu \in \R,~\sigma^2>0 \}$. 
Najskôr budeme testovať hypotézu $H_0$ proti alternatíve $H_1$, ktoré budú tvaru 
$$ H_0: \mu = 0,~H_1: \mu \neq 0. $$
Jednovýberovým testom budeme testovať vygenerované dáta z normovaného normálneho rozdelenia. 
Tento postup budeme opakovať niekoľko krát po sebe. 
Potrebujeme zistiť počet zamietnutých platných hypotéz. 
Keďže dáta sme vygenerovali tak, aby ich stredná hodnota bola rovná $0$, 
všetky hypotézy sú platné a podiel zamietnutých hypotéz ku všetkým opakovaniam bude rovný hladine testu. 

\begin{figure}[h!]
  \centering
  \includegraphics[width=\linewidth]{C:/Users/KIKA/Desktop/BP/R/2hladina1.png}
  \caption{Hladina jednovýberového testu po $1000$ opakovaniach pre rôzne počty pozorovaní}
  \label{obr02:01}
\end{figure}

Na obrázku \ref{obr02:01} môžeme vidieť hladinu jednovýberového testu, ktorým sa testuje stredná hodnota dát s rôznym počtom pozorovaní, 
postupne $10$, $100$, $1000$, $10000$, $100000$. 
Jednovýberový test bol opakovaný $1000$-krát, pričom dáta boli vždy nanovo vygenerované. 
Na obrázku je vidieť, že hladina testu pri rôznych počtoch pozorovaní sa pohybuje blízko zvolenej hladiny testu. 

Teraz budeme postupovať podobne, ale budeme testovať 3 hypotézy. 
Predpokladajme, že máme 3 náhodné výbery. 
Pre jednoduchosť budeme predpokladať, že náhodné výbery majú rovnaký rozptyl. 
$$ {\bf X}_1 = (X_{11}, X_{12}, \dots, X_{1n}) \sim N(\mu_1, \sigma^2) $$ 
$$ {\bf X}_2 = (X_{21}, X_{22}, \dots, X_{2n}) \sim N(\mu_2, \sigma^2) $$
$$ {\bf X}_3 = (X_{31}, X_{32}, \dots, X_{3n}) \sim N(\mu_3, \sigma^2) $$
Tieto dáta budú opäť vygenerované z normovaného normálneho rozdelenia. 
Tentokrát budeme testovať, či sa rovnajú stredné hodnoty všetkých náhodných výberov
a použijeme k tomu mnohonásoné testovanie. 
Budeme mať 3 hypotézy. 
$$ H_0: \mu_1 = \mu_2,~H_1: \mu_1 \neq \mu_2 $$ 
$$ H_0: \mu_2 = \mu_3,~H_1: \mu_2 \neq \mu_3 $$ 
$$ H_0: \mu_1 = \mu_3,~H_1: \mu_1 \neq \mu_3 $$ 

Každú nulovú hypotézu otestujeme použitím jednovýberového testu. 
Mnohonásobné testovanie zopakujeme viackrát po sebe. 
Dáta boli vygenerované so strednou hodnotou rovnou $0$, 
teda všetky náhodné výbery majú rovnakú strednú hodnotu a hypotézy sú platné. 
Stačí spočítať všetky zamietnuté hypotézy a ich podiel k počtu opakovaní testovania je hladina testu. 

\begin{figure}[h!]
  \centering
  \includegraphics[width=\linewidth]{C:/Users/KIKA/Desktop/BP/R/2hladina2.png}
  \caption{Hladina mnohonásobného testovania po $1000$ opakovaniach pre rôzne počty pozorovaní}
  \label{obr02:02}
\end{figure}

Na obrázku \ref{obr02:02} vidíme hladinu mnohonásobného testovania, ktorým sa testuje rovnosť stredných hodnôt všetkých náhodných výberov. 
Opäť boli použité rôzne počty pozorovaní, postupne $10$, $100$, $1000$, $10000$, $100000$. 
Mnohonásobné testovanie bolo opakované $1000$-krát, dáta boli vždy vygenerované nanovo. 
Na obrázku vidíme, že skutočná hladina testu je vzdialená od $\alpha$, ktorú sme si zvolili predtým. 
Jej veľkosť je skoro trojnásobkom zvolenej hladiny testu. 

Hladina testu je v tomto prípade výrazne vyššia ako by sme chceli, 
preto nie je možné používať mnohonásobné testovanie v takejto forme. 




\begin{table}[h!]
  \centering
  \begin{tabular}{|c|r@{.}l|r@{.}l|r@{.}l|r@{.}l|r@{.}l|r@{.}l|}
    \hline
     & \multicolumn{2}{c}{BONF} & \multicolumn{2}{c}{ŠIDÁK} & \multicolumn{2}{c}{HOLM} 
     & \multicolumn{2}{c}{SIMES} & \multicolumn{2}{c}{HOCH} & \multicolumn{2}{c}{BH} \\ \hline
    1 & 0&23 & 0&15 & 5 & 6\\ \hline
    2 & 0&2 & 0&155 & 5 & 6\\ \hline
    3 & \\ \hline
  \end{tabular}
  \caption{Počet zamietnutých hypotéz v každej metóde pri testovaní $3$ hypotéz 
  pri počte opakovaní $10000$}
  \label{tab04:}
\end{table}